\section{Units of a number field}

To use Minkowsky-theory results to talk about units in $\calO_K$, we need to translate our arguments to a multiplicative setting.

\subsection{Taking logarithms}

Recall the canonical mapping $j:K\to K_\bbC$ taking $a\in K$ to the $(r+2s)$-dimensional vector $(\tau a)$. Note that if we restrict the domain to $K^*$ (the unit field of $K$), it is clear that embeddings into $\bbC$ map units to units, so we get a homomorphism
\[
	j : K^* \to K^*_\bbC.
\]

Following a similar construction to that of the trace, we define the `norm' homomorphism
\[
	\N : K^*_\bbC \to \bbC^*
\]
given by multiplying the coordinates. Again, taking the composite
\begin{center}
	\begin{tikzcd}
		K^* \arrow[r,"j"] & K^*_\bbC \arrow[r,"\N"] & \bbC^*
	\end{tikzcd}
\end{center}
gives the usual norm of $K\mid\bbQ$, i.e. $\N_{K\mid\bbQ}(a)=\N(ja)$. The following diagram represents the bulk of our work:
\begin{center}
	\begin{tikzcd}
		K^* \arrow[d,"\N_{K\mid\bbQ}"'] \arrow[r,"j"] & K^*_\bbC \arrow[d,"\N"] \arrow[r,"l"] & \prod_\tau \bbR \arrow[d,"\Tr"]\\
		\bbQ^* \arrow[r,"\subset"] & \bbC^* \arrow[r,"l"] & \bbR.
	\end{tikzcd}
\end{center}

The most logical choice of a homomorphism from $K^*_\bbC$ (a multiplicative group) to $\prod_\tau\bbR$ (an additive group) is to apply the logarithm. We define $l:\bbC^*\to\bbR$ by $l(z)=\log|z|$, and this of course extends to a surjective $l:K^*_\bbC\to\prod_\tau\bbR$ by applying the logarithm on each coordinate. It is routine and requires no tricks to check that the diagram commutes.

Now, recall that the involution $F$, an element of the Galois group $G(\bbC\mid\bbR)$, acts trivially on $K^*$, on $K^*_\bbC$ as we defined in the previous section, and on any $(x_\tau)\in\prod_\tau\bbR$ by $(Fx)_\tau = x_{\overline{\tau}}$. As before, we restrict our attention to $G(\bbC\mid\bbR)$-invariant subgroups of the diagram above, and we get the following diagram:

\begin{center}
	\begin{tikzcd}
		K^* \arrow[d,"\N_{K\mid\bbQ}"'] \arrow[r,"j"] & K^*_\bbC \arrow[d,"\N"] \arrow[r,"l"] & \left[\prod_\tau \bbR\right]^+ \arrow[d,"\Tr"]\\
		\bbQ^* \arrow[r,"\subset"] & \bbR^* \arrow[r,"l"] & \bbR.
	\end{tikzcd}
\end{center}

\subsection{Dirichlet's unit theorem}

First of all, we can be sure that $\calO^*_K$ contains the group of roots of unity in $K$, which is of course finite; we denote it by $\mu(K)$. Indeed, if $a\in K$ is a root of unity, its norm $\N_{K\mid\bbQ}(a)$ is the product of all the roots of unity, which by Vieta equals $\pm 1$ times the constant term of $x^n-1=0$. Hence $\N_{K\mid\bbQ}(a)=\pm 1$, so $a$ is a unit. It is clear that roots of unity, by the polynomial they solve, are integral over $K$ and thus $\mu(K)$ lies in $\calO^*_K$. This is all we can say about the units of a number field thus far.

Consider the following subgroups of the upper part of the previous diagram:
\begin{itemize}
	\item $\calO^*_K = \{\epsilon\in\calO_K \mid \N_{K\mid\bbQ}(\epsilon) = \pm 1\}$, the group of units of $\calO_K$,
	\item $S=\{y\in K^*_\bbR \mid \N(y)=\pm 1\}$, the ``norm-one surface'', and
	\item $H=\{x\in\left[\prod_\tau\bbR\right]^+ \mid \Tr(x)=0\}$, the ``trace-zero hyperplane''.
\end{itemize}

These restrictions give homomorphisms
\begin{center}
	\begin{tikzcd}
		\calO^*_K \arrow[r,"j"] & S \arrow[r,"l"] & H,
	\end{tikzcd}
\end{center}
the composite of which, $\lambda := l\circ j$, gives an image that we denote by
\[
	\Gamma = \lambda(\calO_K) \subseteq H.
\]

\begin{prop}[Neukirch 7.1]
The sequence
\begin{center}
	\begin{tikzcd}
		1 \arrow[r] & \mu(K) \arrow[r,"\subset"] & \calO^*_K \arrow[r,"\lambda"] & \Gamma \arrow[r] & 0
	\end{tikzcd}
\end{center}
is exact.
\end{prop}
\begin{proof}
	Fix $\xi\in\mu(K)$. We have
	\begin{align*}
		\lambda(\xi) = l\circ j(\xi) &= \left(\log|\tau\xi|\right)_\tau\\
			&= (\log 1) = 0,
	\end{align*}
	so $\mu(K)\subseteq\ker\lambda$.

	Now fix $\epsilon\in\calO^*_K$. We have $\lambda(\epsilon)=0$, so $\log|\tau\epsilon|=0$ for each embedding $\tau$, i.e. $|\tau\epsilon|=1$. This implies that $j\epsilon$ is bounded in $K_\bbR$, but bounded regions of the lattice $j\calO_K$ contain finitely many lattice points, so $\ker\lambda$ is finite. Since it also contains $\mu(K)$ and is a group under multiplication, it is clear that $\epsilon$ is necessarily a root of unity.

	Hence $\ker\lambda\subseteq\mu(K)$, and thus $\ker\lambda=\mu(K)$.
\end{proof}

The following finiteness result will allow us to show that $\Gamma$ is a (complete) lattice in $H$.

\begin{lem}[Neukirch 7.2]
	Up to multiplication by units, there are only finitely many elements $\alpha\in\calO_K$ of given norm $\N_{K\mid\bbQ}(\alpha)=a$.
\end{lem}
\begin{proof}
	Fix $a\in\bbZ$, $a>1$. We claim that in every coset of of $\calO_K/a\calO_K$ there is at most (up to units) one element $\alpha$ with $|\N(\alpha)|=|\N_{K\mid\bbQ}(\alpha)|=a$. Else, if $\beta=\alpha+a\gamma$ is another ($\gamma\in\calO_K$), then
	\[
		\alpha/\beta = 1 \pm \N(\beta)/\beta \gamma \in \calO_K
	\]
	since $\N(\beta)/\beta \in \calO_K$, and similarly $\beta/\alpha\in\calO_K$, so $\beta$ is associated to $\alpha$.
\end{proof}

\begin{thm}[Neukirch 7.3]
	The group $\Gamma$ is a complete lattice in the $(r+s-1)$-dimensional vector space $H$, and is therefore isomorphic to $\bbZ^{r+s-1}$.
\end{thm}
\begin{proof}
	Neukirch's proof proceeds as follows: first, it is shown that $\Gamma$ is indeed a lattice by considering the mapping from which $\lambda:\calO_K^*\to H$ arises by restriction. To show that $\Gamma$ is complete, the proof invokes Minkowski's lattice theorem to construct a bounded set that spans the lattice (in this case, through multiplicative translates).
\end{proof}

Having established that $\Gamma$ has a nice $\bbZ$-module structure, understanding $\calO^*_K$ requires very little effort.

\begin{thm}[Dirichlet's unit theorem -- Neukirch 7.4]
	The group of units $\calO^*_K$ of $\calO_K$ is the direct product of the finite cyclic group $\mu(K)$ and a free abelian group of rank $r+s-1$.
\end{thm}
\begin{proof}
	Recall the exact sequence
	\begin{center}
		\begin{tikzcd}
			1 \arrow[r] & \mu(K) \arrow[r,"\subset"] & \calO^*_K \arrow[r,"\lambda"] & \Gamma \arrow[r] & 0.
		\end{tikzcd}
	\end{center}
	Let $v_1,\dots,v_t$ be a $\bbZ$-basis of $\Gamma$, and $\epsilon_1,\dots,\epsilon_t\in\calO^*_K$ be preimages of the $v_i$. Denote by $A$ the subgroup of $\calO^*_K$ generated by the $\epsilon_i$, and note that $\mu(K)\cap A=\{1\}$, for $\lambda(\epsilon_i)\neq 0$ except for the trivial case. Hence $A$ maps isomorphically onto $\Gamma$ through $\lambda$, so $\calO^*_K=\mu(K)\times A$.

	It is clear that $A$ is of rank $r+s-1$ over $\calO_K$.

\end{proof}

All this work shows that any unit $\epsilon\in\calO^*_K$ can be written uniquely as
\[
	\epsilon = \xi \epsilon_1^{v_1}\cdots\epsilon_t^{v_t},
\]
where $\xi$ is a root of unity and $\epsilon_i$ the ``fundamental'' units of $A$.


\subsection{Example -- Pell's equation}

\begin{prop}[Neukirch, Exercise 5.1]
	Let $D>1$ be a square-free integer and $d$ the discriminant of the real quadratic number field $K=\bbQ(\sqrt{D}))$. Let $x_1,y_1$ be the uniquely determined rational integer solution of one of the equations
	\[
		x^2-dy^2=\pm 4,
	\]
	with $x_1,y_1$ positive and as small as possible. Then
	\[
		\epsilon_1 = \frac{x_1+y_1\sqrt{d}}{2}
	\]
	is a fundamental unit of $K$.
\end{prop}
\begin{proof}
	First, note that $\{1,\sqrt{d}\}$ serves as a basis for $\bbQ(\sqrt{D})$ as a $\bbQ$-vector space. The key behind attacking Pell's equation lies in the equality
	\[
		x^2-dy^2 = (x+y\sqrt{d})(x-y\sqrt{d}) = \N_{K\mid\bbQ}(x+y\sqrt{d}).
	\]
	The goal is to show that an integral solution $(x_1,y_1)$ to Pell's equation does indeed give rise to an algebraic integer of the form of $\epsilon_1$, which is furthermore a unit, moreso \emph{fundamental}. The middle step is the easiest part of the proof; the rest relies in case-work depending on the residue of $D$ modulo 4. It is significant, and once more confirmation of the complexity of integer rings (or even more fundamentally, of integers), that one cannot easily avoid case-work here.

	In what follows, we let $(x_1,y_1)$ be the smallest pair of positive integers solving
	\[
		x_1^2-dy_1^2 = \pm 4.
	\]

	Exercise 5 in section 2 of Neukirch's text gives integral bases for all quadratic extensions of $\bbQ$. We readily use this result. Let $d$ be the discriminant of $K=\bbQ(\sqrt{D})$.

	For $D\equiv 1 \pmod{4}$, we have $d=D$ and an integral basis $\{1,\frac{1}{2}(1+\sqrt{d})\}$. If $x_1$ is even, $dy_1^2$ has to be even for Pell's equation to hold, but $d$ is odd, so $y_1$ is even as well. If $x_1$ is odd, $dy_1^2$ has to be odd for the same reason. In any case, we conclude that $x_1$ and $y_1$ must have the same parity, so their difference $x_1-y_1$ must be even. Note that
	\begin{align*}
		\epsilon_1 = \frac{x_1+y_1\sqrt{d}}{2} &= \frac{x_1-y_1+y_1+y_1\sqrt{d}}{2}\\
			&= \frac{x_1-y_1}{2} + y_1\frac{1}{2}\left(1+\sqrt{d}\right),
	\end{align*}
	so $\epsilon_1\in\calO_K$.

	For $D\equiv 2,3 \pmod{4}$, we have $d=4D$ and an integral basis $\{1,\frac{1}{2}\sqrt{d}\}$. Substituting $d$ in Pell's equation, we conclude that $x_1$ is even. Hence
	\[
		\epsilon_1 = \frac{x_1+y_1\sqrt{d}}{2} = \frac{x_1}{2} + y_1\frac{\sqrt{d}}{2},
	\]
	so $\epsilon_1\in\calO_K$.

	In any case, to check the norm of $\epsilon_1$ we need not leave $K$:
	\begin{align*}
		\N_{K\mid\bbQ}(\epsilon_1) &= \N_{K\mid\bbQ}\left(\frac{1}{2}\right)\N_{K\mid\bbQ}\left(x_1+y_1\sqrt{d}\right)\\
			&= \frac{\pm 4}{4} = \pm 1,
	\end{align*}
	so $\epsilon_1$ is indeed a unit.

	We now invoke Dirichlet's unit theorem: it tells us that $\calO^*_K = \mu(K)\times \bbZ^n$, with $n=\#\text{real embeddings}+\#\text{complex embeddings}-1$, in this case $n=2+0-1=1$ so $\calO^*_K = \mu(K)\times\bbZ$, but $\bbQ(\sqrt{D})\subset \bbR$ so $\mu(K)=\{\pm 1\}$, which shows that $\epsilon_1$ is \emph{not} a root of unity (for $y_1>0$), and simplifies the free part of the unit group.

	For the final part, let $u=u_1+u_2\sqrt{d}$ be a fundamental unit of our ring of integers $\calO_K$ (i.e. a generator of $\calO^*_K$), so that $(u_1,u_2)$ is a solution to Pell's equation, and assume for the sake of contradiction that $\epsilon_1\neq u$, so in particular $\epsilon_1=u^k$ for some $k>0$. We will use a monotoneness result regarding powers of numbers of the form of $\epsilon_1$. Fix $a$ and $b$ positive integers, and let $\beta = \frac{a+b\sqrt{d}}{2} = \frac{a}{2} + \frac{b}{2}\sqrt{d}$. In general, we have
	\[
		\beta^2 = \frac{a^2+db^2+2ab\sqrt{d}}{4} = \frac{a^2+db^2}{4} + \frac{ab}{2}\sqrt{d}.
	\]

	In any congruence case, the first term of the power $\beta^2$ is strictly greater than the first term of $\beta$. From this, we conclude that the first term of $u$ is necessarily smaller than the first term of $\epsilon_1$, i.e. smaller than $x_1$, which contradicts the minimality assumption.

\end{proof}
