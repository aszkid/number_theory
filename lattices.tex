\section{Minkowski theory}

A good amount of fundamental questions in number theory (including the finiteness result of the class group) can be answered through the study of certain geometric aspects of number fields. To be more specific, one considers points of an $n$-dimensional algebraic number field $K\mid\bbQ$ as points in $n$-dimensional complex space.

\subsection{Lattices}

\begin{defn}
	Let $V$ be an $n$-dimensional $\bbR$-vector space. A \emph{lattice} in $V$ is a subgroup of the form
	\[
		\Gamma = \bbZ v_1 + \cdots + \bbZ v_m
	\]
	with linearly independent vectors $v_1,\dots,v_m\in V$. The $m$-tuple $(v_1,\dots,v_m)$ is called a \emph{basis} for the lattice and the set
	\[
		\Phi = \{x_1v_1 + \cdots + x_mv_m \mid x_i\in\bbR, 0\leq x_i< 1\}
	\]
	a \emph{fundamental mesh} of the lattice. It is called \emph{complete} or a \emph{$\bbZ$-structure} of $V$ if $m=n$.
\end{defn}

Note that a complete lattice is `complete' in the sense that translates of $\Phi$ cover all of $V$.

For our applications to number fields, we prefer to characterize lattices without referring to `linearly independent' vectors. To begin with, lattices are finitely generated subgroups of $V$, but not necessarily the other way around (e.g. $\bbZ+\bbZ\sqrt{2}$ is a subgroup of $V=\bbR$ but not a lattice, since 1 and $\sqrt{2}$ are not linearly independent). The full characterization considers lattices $\Gamma=\bbZ v_1 + \cdots + \bbZ v_m$ as \emph{discrete subgroups} of $V$: every point $\gamma\in\Gamma$ must have a neighborhood in which no other point of the lattice is contained, i.e. it must be isolated.

\begin{prop}
	A subgroup $\Gamma\subseteq V$ is a lattice if and only if it is discrete.
\end{prop}
\begin{proof}
	($\Longrightarrow$) Fix $\gamma\in\Gamma$, so that $\gamma=a_1v_1+\cdots+a_mv_m$ with $a_i\in\bbZ$. Extend the basis of $\Gamma$ to a basis of $V$; then the set
	\[
		\{x_1v_1 + \cdots + x_nv_n \mid x_i\in\bbR, |x_i-a_i|<1\}
	\]
	is a neighborhood in which $\gamma$ is isolated. Hence $\Gamma$ is discrete.

	($\Longleftarrow$) We first show that $\Gamma$ being a discrete subgroup of $V$ implies that it is closed: for choose any neighborhood $U$ of 0, then there is a smaller neighborhood $U'\subseteq U$ of 0 such that any difference of vectors in $U'$ is in $U$ (just make $U'$ small enough). If there were some $x\notin\Gamma$ in the closure of $\Gamma$, then $x+U'$ would contain a difference $\gamma_1-\gamma_2\in x+U'$ of distinct lattice points $\gamma_1,\gamma_2\in\Gamma$, but then $0\neq \gamma_1-\gamma_2\in U$, so the arbitary neighborhood $U$ around 0 contains the lattice point $\gamma_1-\gamma_2$, i.e. 0 is not isolated, so $\Gamma$ is not a discrete subgroup of $V$, a contradiction. Hence $\Gamma$ is closed.

	We now show that $\Gamma$ can be given a basis in the lattice sense, with linearly independent vectors. Let $V_0$ the be the linear subspace of $V$ spanned by $\Gamma$, and let $m$ be its dimension. Choose a basis $u_1,\dots,u_m\in\Gamma$ for $V_0$, and consider the complete lattice $\Gamma_0 = \bbZ u_1 + \cdots + \bbZ u_m \subseteq\Gamma$ of $V_0$. We claim that $q=(\Gamma:\Gamma_0)$ is finite: let $\gamma_i\in\Gamma$ vary over a system of representatives of the cosets in $\Gamma/\Gamma_0$. Since $\Gamma_0$ is complete, translates of its fundamental mesh $\Phi_0$ cover $V_0$. Since $\Gamma\cap\overline{\Phi_0}$ is compact and discrete, it is finite, which implies that the number of representatives $\gamma_i$ is finite.

	For any coset $\gamma+\Gamma_0$ in the quotient we have $q(\gamma+\Gamma_0)=\Gamma_0$, but $q(\gamma+\Gamma_0)=q\gamma+\Gamma_0$, so $q\gamma\in\Gamma_0$. The choice of $\gamma\in\Gamma$ was arbitrary, so $q\Gamma\subseteq\Gamma_0$. Hence
	\[
		\Gamma\subseteq\frac{1}{q}\Gamma_0 = \bbZ\left(\frac{1}{q}u_1\right) + \cdots + \bbZ\left(\frac{1}{q}u_m\right).
	\]

	By classification of finitely generated abelian groups, $\Gamma$ admits a $\bbZ$-basis $\{v_i\}$ of dimension $r\leq m$. But recall that $\Gamma$ spans $V_0$, which is $m$-dimensional, so $r=m$ and the $v_i$ are linearly independent.
\end{proof}

The following lemma is a criterion for identifying complete lattices.

\begin{lem}
	A lattice $\Gamma$ in $V$ is complete if and only if there exists a bounded subset $M\subseteq V$ such that the collection of all translates $M+\gamma$, with $\gamma\in\Gamma$, covers all of $V$.
\end{lem}
\begin{proof}
	($\Longrightarrow$) Given that $\Gamma=\bbZ v_1 + \cdots + \bbZ v_n$ is complete, we let $M$ be the fundamental mesh of the lattice. Translates of this, by definition, covers $V$.

	($\Longleftarrow$) Given such a bounded $M\subset V$, whose translates cover $V$, let $V_0$ be the subspace spanned by $\Gamma$. Let $v\in V$, so that for any $n\in\bbN$ we have
	\[
		nv = a_n + \gamma_n
	\]
	with $a_n\in M$ and $\gamma_n\in\Gamma\subset V_0$.

	By $M$ being bounded, $\frac{1}{n}a_n \to 0$ as $n\to\infty$, and by $V_0$ being closed we get
	\begin{align*}
		v &= \lim_{n\to\infty}\frac{1}{n}a_n + \lim_{n\to\infty}\frac{1}{n}\gamma_n\\
			&= \lim_{n\to\infty} \frac{1}{n}\gamma_n,
	\end{align*}
	which lies in $V_0$.
\end{proof}

If we restrict our attention to $V$ an euclidean vector space, we have a sensible notion of volume, which we denote by $\vol$.

A subset $X$ of $V$ is \emph{centrally symmetric} if, given any $x\in X$, the point $-x$ is also in $X$. The subset is \emph{convex} if the line segment joining two points $x,y\in X$ is contained in $X$.

\begin{thm}[Minkowski's Lattice Point Theorem]
	Let $\Gamma$ be a complete lattice in the euclidean vector space $V$ and $X$ a centrally symmetric, convex subset of $V$. Suppose that $\vol(X) > 2^n\vol(\Gamma)$. Then $X$ contains at least one non-zero lattice point of $\Gamma$.
\end{thm}
\begin{proof}
	If we find two distinct lattice points $\gamma_1,\gamma_2\in\Gamma$ such that
	\[
		\left(\frac{1}{2}X + \gamma_1\right)\cap\left(\frac{1}{2}X + \gamma_2\right) \neq \emptyset,
	\]
	we get a pair of points $x_1,x_2\in X$ such that
	\[
		\left(\frac{1}{2}x_1 + \gamma_1\right) = \left(\frac{1}{2}x_2 + \gamma_2\right),
	\]
	so $\gamma=\gamma_1-\gamma_2=\frac{1}{2}x_1-\frac{1}{2}x_2$ is a lattice point, and the midpoint between $x_1$ and $-x_2$ (which is in $X$ by central symmetry), so $\gamma\in X$ by $X$ being convex. Hence $\gamma\in\Gamma\cap X$.


	Assume that the two sets above were disjoint for any pair of lattice points; then $\Phi\cap\left(\frac{1}{2}X+\gamma\right)$ are also pairwise disjoint for any fundamental mesh $\Phi$, so 
	\[
		\vol(\Phi) \geq \sum_{\gamma\in\Gamma} \vol\left(\Phi\cap\left(\frac{1}{2}X+\gamma\right)\right).
	\]

	But translating $\Phi\cap\left(\frac{1}{2}X+\gamma\right)$ by $-\gamma$ gives $(\Phi-\gamma)\cap\frac{1}{2}X$, which has equal volume, and $\Phi-\gamma$ ranging over $\Gamma$ cover all of $V$, hence also $\frac{1}{2}X$. Therefore
	\begin{align*}
		\vol(\Phi) \geq \sum_{\gamma\in\Gamma} \vol\left(\Phi\cap\left(\frac{1}{2}X+\gamma\right)\right) &= \sum_{\gamma\in\Gamma}\vol\left((\Phi-\gamma)\cap\frac{1}{2}X\right)\\
			&= \vol\left(\frac{1}{2}X\right)\\
			&= \frac{1}{2^n}\vol(X),
	\end{align*}
	a contradiction.
\end{proof}

In other words; a `nice' (centrally symmetric, convex), `large enough' (volume condition) subset of $V$ always contains a lattice point.


\subsection{Setup in number fields}

In a nutshell, we will construct a vector space $K_\bbR$ from an arbitrary number field $K\mid\bbQ$ that will turn out to be isomorphic to a real euclidean vector space. Minkowski's lattice point theorem will apply, which will prove useful as a tool in our final result.

We start by defining the $\bbC$-vector space $K_\bbC := \prod_\tau \bbC$ as the product of the complex plane indexed by all $n$ complex embeddings $\tau:K\to\bbC$, and the canonical map $j:K\to K_\bbC$ sending each $a\in K$ to the vector $(\tau a)$.

Remember that the Galois group $G(\bbC\mid\bbR)$ is generated by complex conugation $F:z\mapsto\overline{z}$. This conjugation map acts on factors of $\prod_\tau\bbC$ and on the indexes themselves, by taking $\tau\mapsto\overline{\tau}$. In particular, we get the involution $F:K_\bbC\to K_\bbC$ that takes points $z=(z_\tau)\in K_\bbC$ to $(Fz)_\tau = \overline{z}_{\overline{\tau}}$.

Finally, we define a trace $\Tr:K_\bbC\to\bbC$ by $\Tr(z) = \sum_\tau z_\tau$, which is again equivariant under $F$. Note that the composition of maps
\begin{center}
	\begin{tikzcd}
		K \arrow[r,"j"] & K_\bbC \arrow[r,"\Tr"] & \bbC
	\end{tikzcd}
\end{center}

gives $\Tr\circ j(a) = \Tr((\tau a)_\tau) = \sum_\tau \tau a$, the usual trace $\Tr_{\bbC\mid\bbQ}$.

We now restrict our attention to the real vector space $K_\bbR = K_\bbC^+ = \left[\prod_\tau\bbC\right]^+$, consisting of the $G(\bbC\mid\bbR)$-invariant, i.e. conjugation-invariant points of $K_\bbC$; that is, $(z_\tau)\in K_\bbC$ for which $z_{\overline{\tau}}=\overline{z}_\tau$. Indeed, for such a point we have $(\overline{z})_{\overline{\tau}}= \overline{\overline{z}}_\tau = z_\tau$. Hence for $a\in K$, we have $F(ja)=ja$, so $j$ restricts to
\[
	j : K\to K_\bbR.
\]

We call the euclidean space $K_\bbR$ the \emph{Minkowski space} of $K$, its scalar product the \emph{canonical metric}, and the associated Haar measure (volume function) the \emph{canonical measure}. The trace restricts as well to the usual trace of $K\mid\bbQ$.

Of all the embeddings $\tau:K\to\bbC$, some have a purely real image, which we denote by
\[
	\rho_1,\dots,\rho_r:K\to\bbR,
\]
and the complex ones are given in pairs (by conjugation),
\[
	\sigma_1,\overline{\sigma_1},\dots,\sigma_s,\overline{\sigma_s}:K\to\bbC.
\]

Hence, we get 
\[
	K_\bbR = \{(z_\tau)\in\prod_\tau\bbC \mid z_\rho\in\bbR, z_{\overline{\sigma}} = \overline{z}_\sigma\}.
\]


\begin{prop}
	There is an isomorphism
	\[
		f : K_\bbR \to \prod_\tau \bbR = \bbR^{r+2s}
	\]
	given by $(z_\tau)\mapsto(x_\tau)$, where
	\begin{align*}
		x_\rho = z_\rho, x_\sigma = \Re(z_\sigma), x_{\overline{\sigma}} = \Im(z_\sigma).
	\end{align*}
\end{prop}
\begin{proof}
	By our previous discussion, this map is an isomorphism.
\end{proof}

The two different measures (on $K_\bbR$ and $\bbR^{r+2s}$) differ by
\[
	\vol_{\text{canonical}}(X) = 2^s\vol_{\text{Lebesgue}}(f(X)).
\]


The map $j:K\to K_\bbR$, which `embeds' the number field $K$ into the euclidean vector space $K_\bbR$, is the beginning of our study of `geometric' number theory.

\begin{prop}
	If $\fraka\neq 0$ is an ideal of $\calO_K$, then $\Gamma=j\fraka$ is a complete lattice in $K_\bbR$. Its fundamental mesh has volume
	\[
		\vol(\Gamma) = \sqrt{|d_K|}(\calO_K:\fraka).
	\]
\end{prop}
\begin{proof}
	Let $\alpha_1,\dots,\alpha_n$ be a $\bbZ$-basis for $\fraka$, i.e.
	\begin{align*}
		\Gamma = j\fraka &= j\left(\bbZ\alpha_1+\cdots+\bbZ\alpha_n\right)\\
			&= \bbZ j\alpha_1 + \cdots + \bbZ j\alpha_n,
	\end{align*}
	so $j\fraka$ is a lattice in $K_\bbR$. To see that it is complete, notice that

	Enumerate all the embeddings $\tau_i:K\to\bbC$, and we have a matrix $A=(\tau_i\alpha_j)$, which relates to the discriminant of $\fraka$ by
	\begin{align*}
		d(\fraka) = d(\alpha_1,\dots,\alpha_n) &= (\det A)^2\\
			&= (\calO_K:\fraka)^2 d_K,
	\end{align*}
	and
	\begin{align*}
		(\langle j\alpha_i, j\alpha_k\rangle) &= (\sum_{l=1}^n \tau_l\alpha_i \overline{\tau}_l\alpha_k)\\
			&= A\overline{A}^t,
	\end{align*}
	so 
	\begin{align*}
		\vol(\Gamma) &= \left|\det(\langle j\alpha_i,j\alpha_k\rangle)\right|^{1/2}\\
			&= \sqrt{|d_K|}(\calO_K:\fraka).
	\end{align*}
\end{proof}

The following result, chiefly using Minkowski's lattice theorem, is the fundamental tool in our next results on $Cl_K$.

\begin{thm}
	Let $\fraka\neq 0$ be an integral ideal of $K$, and let $c_\tau>0$, for $\tau\in\Hom(K,\bbC)$, be real numbers such that $c_\tau=c_{\overline{\tau}}$ and
	\[
		\prod_\tau c_\tau > A(\calO_k:\fraka),
	\]
	where $A=\left(\frac{2}{\pi}\right)^s\sqrt{|d_K|}$. Then there exists $a\in\fraka$, $a\neq 0$, such that
	\[
		|\tau a| < c_\tau
	\]
	for all $\tau\in\Hom(K,\bbC)$.
\end{thm}
\begin{proof}
	Let $X=\{(z_\tau)\in K_\bbR \mid |z_\tau| < c_\tau\}$. We claim that this set is centrally symmetric and convex. We bring $X$ over to euclidean space to compute its volume more easily, through the isomorphism we have defined in (??). Hence its volume is $2^s$ times the Lebesgue volume of the image
	\[
		f(X) = \{(x_\tau)\in\prod_\tau\bbR \mid |x_\rho| < c_\rho, x_{\sigma}^2+x_{\overline{\sigma}}^2 < c_{\sigma}^2\},
	\]
	i.e.
	\begin{align*}
		\vol(X) = 2^s \vol_{\text{Lebesgue}}(f(X)) &= 2^s\prod_\rho(2c_\rho)\prod_\sigma(\pi c_\sigma^2)\\
		&= 2^{r+s}\pi^s \prod_\tau c_\tau\\
		&> 2^{r+s}\pi^s \left(\frac{2}{\pi}\right)^s\sqrt{|d_K|}(\calO_K:\fraka)\\
		&= 2^{r+2s}\sqrt{|d_K|}(\calO_K:\fraka) = 2^n\vol(\Gamma),
	\end{align*}

	the last equality given by our previous proposition. Since $\Gamma$ is complete in $K_\bbR$ and $X$ centrally symmetric, convex and with large enough volume, our setup satisfies the conditions of the Minkowski lattice point theorem. This immediately gives a lattice point $\gamma$ in $X$, i.e. some $a\in\fraka$ for which $|\tau a|<c_\tau$ for all $\tau\in\Hom(K,\bbC)$.
\end{proof}



\subsection{Class number}

We want to show that $Cl_K=J_K/P_K$ is finite, i.e. that the number of classes of fractional ideals that differ by a multiple of a principel ideal is finite. Rough outline: show that under some sensible notion of the `norm' of an ideal, only finitely many ideals of $\calO_K$ are bounded above by some number. Then show that each class of $Cl_K$ has an integral ideal satisfying the same bound. Hence there are only finitely many classes, i.e. $Cl_K$ is finite.

Given an integral ideal $\fraka$, denote its \emph{absolute norm} by
\[
	\frakN(\fraka) = (\calO_K:\fraka).
\]

This index is again finite by our previous considerations. In particular, this quantity is deserves its name by considering the case $\fraka=(\alpha)$, for which
\[
	\frakN\left((\alpha)\right) = |N_{K\mid\bbQ}(\alpha)|,
\]
sine a $\bbZ$-basis $\omega_1,\dots,\omega_n$ of $\calO_K$ gives a $\bbZ$-basis $\alpha\omega_1,\dots,\alpha\omega_n$ of $(\alpha)=\alpha\calO_K$, and the change of basis matrix $A$ gives $|\det(A)|=(\calO_K:(\alpha))$, but $\det(A)=N_{K\mid\bbQ}(\alpha)$ too by definition.

We now show that the absolute norm is multiplicative over prime ideals.

\begin{prop}
	If $\fraka=\frakp_1^{v_1}\cdots\frakp_r^{v_r}$ is the prime factorization of an ideal $\fraka\neq 0$, then we have
	\[
		\frakN(\fraka) = \frakN(\frakp_1)^{v_1}\cdots\frakN(\frakp_r)^{v_r}.
	\]
\end{prop}
\begin{proof}
	By the Chinese remainder theorem, we have
	\[
		\calO_k/\fraka = (\calO_k/\frakp_1)^{v_1}\oplus\cdots\oplus(\calO_k/\frakp_r)^{v_r}.
	\]
	
	Hence the absolute norm of any $\fraka$ is the product of powers of the norms of its prime factors. We thus consider only the case where $\fraka=\frakp^v$.

	Note that the chain
	\[
		\frakp \supseteq \frakp^2 \supseteq \cdots \supseteq \frakp^v
	\]
	is proper, in the sense that $\frakp^i\neq\frakp^{i+1}$, by unique factorization. Furthermore, each quotient $\frakp^i/\frakp^{i+1}$ is a 1-dimensional $\calO_K/\frakp$-vector space: fix $a\in\frakp^i\setminus \frakp^{i+1}$ and let $\frakb=(a)+\frakp^{i+1}$. Then $\frakp^i\supseteq\frakb\supset\frakp^{i+1}$, and since $\frakb\frakp^{-i}$ cannot be a proper divisor of $\frakp=\frakp^{i+1}\frakp^{-i}$, for $\frakp$ is prime, we get that $\frakb=\frakp^i$, i.e. $a\pmod{\frakp^{i+1}}$ is a basis of $\frakp^i/\frakp^{i+1}$ as a 1-dimensional $\calO_K/\frakp$-vector space. Hence $\calO_K/\frakp\cong\frakp^i/\frakp^{i+1}$, and thus
	\begin{align*}
		\frakN(\frakp^v) = (\calO_K : \frakp^v) &= (\calO_K:\frakp)(\frakp:\frakp^2)\cdots(\frakp^{v-1}:\frakp^v)\\
			&= (\calO_K:\frakp)^v\\
			&= \frakN(\frakp)^v.
	\end{align*}

	The first equality concering indices is given by the following inductive argument. Let $M=M_0\supseteq M_1\supseteq \cdots\supseteq M_v$ be a sequence of modules for which each quotient $M_i/M_{i+1}$ is simple (it contains no non-zero proper submodules), and assume that $M$ is Noetherian and Artinian. Such a sequence takes the name of a \emph{composition sequence}, and by $M$ being Artinian we must have a finite $v$ for which $M_v=0$. In fact, we have (Eisenbud, 2.13) that any composition sequence of $M$ must have the same $v$, which we call its \emph{length}. We claim that, in general,
	\[
		|M| = \prod_{i=1}^{v-1} |M_i/M_{i+1}|.
	\]

	In fact, by the usual quotient homomorphism $M_0\to M_0/M_1$ we have $|M_0|=|M_0/M_1|\cdot|M_1|$. Again $M_1\to M_1/M_2$ gives $|M_1|=|M_1/M_2|\cdot|M_2|$, so $|M_0|=|M_0/M_1|\cdot|M_1/M_2|\cdot|M_2|$. It is clear that our equality follows.

	Our inductive proof is as follows. For the base case, let $v=1$, and consider the quotient homomorphism $M_0\twoheadrightarrow M_0/M_1=M_0$: this trivially gives $|M|=|M_0|$. Now assume that the product equality holds for modules of length up to $v-1$, and let $M$ be a module of length $v$. Any chain $M=M_0\supseteq M_1\supseteq\cdots\supseteq M_v$ quotiented out by $M_v$ gives a module of length $v-1$, for which the theorem holds. Lifting to our original composition sequence, (...)
	
	For this particular setup, we have $M_i/M_{i+1}\cong\frakp^i/\frakp^{i+1}\cong\calO_K/\frakp$.
\end{proof}

This automatically gives that the absolute norm is multiplicative over all ideals, and thus can be extended to a homomorphism
\[
	\frakN : J_K \to \bbR^*_+.
\]

\begin{lem}
	In every ideal $\fraka\neq 0$ of $\calO_K$ there exists a non-zero $a\in\fraka$ such that
	\[
		|N_{K\mid\bbQ}(a)| \leq \left(\frac{2}{\pi}\right)^s\sqrt{|d_K|}\frakN(\fraka).
	\]
\end{lem}
\begin{proof}
	Fix $\epsilon>0$, and choose positive real numbers $c_\tau$ for each $\tau\in\Hom(K,\bbC)$ such that $c_\tau = c_{\overline{\tau}}$ and
	\[
		\prod_\tau c_\tau = \left(\frac{2}{\pi}\right)^s\sqrt{|d_K|}\frakN(\fraka)+ \epsilon.
	\]

	Applying (??), we get some $a\in\fraka$, non-zero, for which $|\tau a|<c_\tau$. Hence
	\begin{align*}
		|N_{K\mid\bbQ}(a)| = \prod_\tau |\tau a| &< \prod_\tau c_\tau\\
			&= \left(\frac{2}{\pi}\right)^s\sqrt{|d_K|}\frakN(\fraka)+ \epsilon
	\end{align*}
	Our choice of $\epsilon$ was arbitrary, so letting $\epsilon\to 0$ we get the desired inequality.
\end{proof}

\begin{thm}
	The ideal class group $Cl_K$ is finite. Its order $h_K=(J_K:P_K)$ is called the \emph{class number} of $K$.
\end{thm}
\begin{proof}
	Let $\frakp\neq 0$ be a prime ideal of $\calO_K$, and $\frakp\cap\bbZ=p\bbZ$. Then, as mentioned before, $\calO_K/\frakp$ is a finite field extension of $\bbZ/p\bbZ$ of degree, say, $f$. Hence $\frakN(\frakp)=p^f$. In the other direction, a fixed $p$ gives only finitely many ideals $\frakp$ such that $\frakp\cap\bbZ=p\bbZ$, for then $\frakp\mid(p)$ and $(p)$ has finitely many prime factors. Hence for any given bound $M$, since there are only finitely many primes $p_i\leq M$, we conclude that there are only finitely many prime ideals of absolute norm $\leq M$. This same result applies to arbitrary integral ideals, by unique factorization and $\frakN$ being multiplicative.

	We only need to show that each class $[\fraka]\in Cl_K$ contains an integral ideal $\fraka_1$ with
	\[
		\frakN(\fraka_1) \leq M = \left(\frac{2}{\pi}\right)^s\sqrt{|d_K|}.
	\]

	In fact, choose any representative $\fraka$ of the class $[\fraka]$, and fix $\gamma\in\calO_K$ non-zero such that $\frakb=\gamma\fraka\inv\subseteq\calO_K$. By our previous lemma, we have some non-zero $\alpha\in\frakb$ for which
	\begin{align*}
		|N_{K\mid\bbQ}(\alpha)| \frakN(\frakb)\inv \leq \left(\frac{2}{\pi}\right)^s\sqrt{|d_K|},
	\end{align*}
	but in particular
	\begin{align*}
		|N_{K\mid\bbQ}(\alpha)| \frakN(\frakb)\inv &= \frakN((\alpha))\frakN(\frakb\inv)\\
			&= \frakN((\alpha)\frakb\inv),
	\end{align*}
	so $\frakN(\alpha\frakb\inv)\leq M$. Letting $\fraka_1=\alpha\frakb\inv=\alpha\alpha\inv\fraka$, we have found an ideal in $[\fraka]$ for which the desired inequality holds. By our previous reasoning there are only finitely many such ideals, and so $Cl_K$ is finite.
\end{proof}
