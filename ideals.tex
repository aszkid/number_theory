\section{Ideal numbers}

Consider the number field $K=\bbQ(\sqrt{-5})$. By 2.4, its ring of integers is given by $\calO_K=\bbZ+\bbZ\sqrt{-5}$. Here, we can write 21 in two ways:
\[
	21 = 3\cdot 7 = (1+2\sqrt{-5})\cdot(1-2\sqrt{-5}).
\]

We claim that these two factorizations are fundamentally different. In particular, we show that each of the terms in the two decompositions is irreducible in $\calO_K$.

If we had $3=\alpha\beta$ as non-units, taking the norm on both sides gives $9=\N_{K\mid\bbQ}(\alpha)\N_{K\mid\bbQ}(\beta)$, which requires the norm of $\alpha$ to be $\pm 3$, i.e. an integral solution to
\[
	\N_{K\mid\bbQ}(\alpha) = \N_{K\mid\bbQ}(x + y\sqrt{-5}) = x^2 +5y^2 = \pm 3,
\]
which does not exist, purely by inspection. A similar argument works to show that $7$, $1+2\sqrt{-5}$ and $1-2\sqrt{-5}$ are also irreducible in $\calO_K$. Furthermore, 3 and 7 are not associated to $1+2\sqrt{-5}$ nor $1-2\sqrt{-5}$. Hence we have two fundamentally distinct factorizations of 21 in terms of irreducible elements.

The bottom line is that unique factorization does \emph{not} hold in the ring of integers of $\bbQ(\sqrt{-5})$, which is bad news. If we want to do arithemtic in $\calO_K$ the same way we do arithmetic in $\bbZ$, unique factorization into primes is a must. Eduard Kummer proposed the idea of ``ideal numbers'', in order to replace the not-so-useful notion of  irreducibles. The goal is to define ``ideal prime numbers'' $\frakp_1$, $\frakp_2$, $\frakp_3$ and $\frakp_4$ for which
\begin{align*}
	3 = \frakp_1\frakp_2 && 7 = \frakp_3\frakp_4 && 1+2\sqrt{-5} = \frakp_1\frakp_3 && 1-2\sqrt{-5} = \frakp_2\frakp_4,
\end{align*}
so that $21 = \frakp_1\frakp_2\frakp_3\frakp_4$, thus recovering unique factorization.

Whatever an ideal number $\fraka$ should be, we care above all about its divisibility properties. In particular, given $a,b,\lambda\in\calO_K$, we should have $\fraka\mid a$ and $\fraka\mid b$ imply $\fraka\mid a\pm b$, and $\fraka\mid a$ imply $\fraka\mid\lambda a$. Also, just like usual numbers, an ideal $\fraka$ should be totally described by its divisors in $\calO_K$, i.e.
\[
	\fraka = \{a\in\calO_K\mid\fraka\mid a\}.
\]

But this set is, almost by definition, an ideal of $\calO_K$: closed under addition and absorbing multiplication by elements of $\calO_K$. Richard Dedekind realized this, and defined ``ideal numbers'' as the ideals of $\calO_K$. Divisibility relations $\fraka\mid a$ are translated by the inclusion $a\in\fraka$; and more generally for two ideals, $\fraka\mid\frakb$ is translated by $\frakb\subseteq\fraka$.

\subsection{Dedekind domains}


\begin{thm}
	The ring $\calO_K$ is noetherian, integrally closed, and every prime ideal $\frakp\neq 0$ is a maximal ideal.
\end{thm}
\begin{proof}
	Let $\fraka$ be an ideal of $\calO_K$; in particular, it is an $\calO_K$-submodule. Since $\calO_K$ is a free $\bbZ$-module, $\fraka$ is necessarily a free $\bbZ$-module by $A$ being a PID and hence noetherian. Thus $\calO_K$ is noetherian.
	
	That $\calO_K$ is integrally closed is given by definition (it is the integral closure of $\bbZ$ in $K$).

	Finally, let $\frakp\neq 0$ be a prime ideal; we want to show that $\frakp$ is maximal, i.e. that $\calO_K/\frakp$ is a field. In fact, note that $\calO_k/\frakp$ is a finite integral domain, hence $\calO_k/\frakp$ is a field.
\end{proof}

\begin{defn}
	A noetherian, integrally closed integral domain in which every non-zero prime ideal is maximal is called a \emph{Dedekind domain}.
\end{defn}

Dedekind domains, as far as the eye can see, are very much like $\bbZ$ in that unique factorization (of ideals into prime ideals) holds. Proving this is the focus of our next few pages.

Consider an arbitrary Dedekind domain $\calO$, and denote by $K$ its field of fractions. Given two ideals $\fraka$ and $\frakb$ of $\calO$, we say that $\fraka$ divides $\frakb$ and write $\fraka\mid\frakb$ if $\frakb\subseteq\fraka$. We define $\fraka+\frakb=\{a+b\mid a\in\fraka, b\in\frakb\}$ and $\fraka\frakb = \{\sum_i a_ib_i \mid a_i\in\fraka, b_i\in\frakb\}$.

To prove unique factorization of ideals, we require two lemmas.

\begin{lem}
	If $\fraka\neq 0$ is an ideal of $\calO$, then there exist non-zero prime ideals $\frakp_1,\dots,\frakp_r$ such that $\fraka\supseteq\frakp_1\cdots\frakp_r$.
\end{lem}
\begin{proof}
	Let $\frakM$ be the set of ideals of $\calO$ which do not satisfy this condition, and assume it is non-empty. We order $\frakM$ by inclusion, and by $\calO$ being noetherian we get a maximal element $\fraka$, which is necessarily not a prime ideal. Hence we have $b_1,b_2\in\calO$ such that $b_1b_2\in\fraka$ but $b_1,b_2\notin\fraka$. Let $\fraka_1 = (b_1) + \fraka$ and $\fraka_2 = (b_2) + \fraka$, so that $\fraka\subset\fraka_1$, $\fraka\subset\fraka_2$ and $\fraka_1\fraka_2\subset\fraka$. By $\fraka$ being maximal, $\fraka_1$ and $\fraka_2$ are not in $\frakM$ so they must contain a product of prime ideals, which implies that $\fraka$ contains one too, a contradicition.
\end{proof}

Fix a prime ideal $\frakp$ of $\calO$ and define
\[
	\frakp\inv = \{x\in K\mid x\frakp\subseteq\calO\}.
\]
Note that $\frakp\inv$ lies in $K$, not in $\calO$. It consists of elements in $K$ that are brought to lie in $\calO$ through multiplication by $\frakp$, i.e. ``fractions'' that have some element of $\frakp$ in the ``denominator''.

\begin{lem}
	For every ideal $\fraka\neq 0$ of $\calO$ we have $\fraka\frakp\inv\neq\fraka$.
\end{lem}
\begin{proof}
	We will first show that $\frakp\inv\neq\calO$. Then, assuming that $\fraka\frakp\inv=\fraka$, we will derive the contradiction $\frakp\inv=\calO$.

	Fix $a\in\frakp$ non-zero; we have a sequence of primes $\frakp_1\cdots\frakp_r\subseteq (a)\subseteq\frakp$ by the previous lemma, with $r$ as small as possible. We claim that one of $\frakp_i$ is contained in $\frakp$: otherwise for every $i$ there would be some $a_i\in\frakp_i\setminus\frakp$ such that $a_1\cdots a_r\in\frakp$, but $\frakp$ is prime, which implies that one of the $a_i$ has to be in $\frakp$, a contradiction. Hence say $\frakp_1\subseteq\frakp$. Since prime ideals are maximal, we have $\frakp_1=\frakp$, so $\frakp\frakp_2\cdots\frakp_r\subseteq(a)\subseteq\frakp$. Then $\frakp_2\cdots\frakp_r\not\subseteq(a)$, so there exists some $b\in\frakp_2\cdots\frakp_r$ with $b\notin a\calO$, i.e. $a\inv b\notin\calO$. But also $b\frakp\subseteq(a)$ so $a\inv b\frakp\subseteq\calO$, and thus $a\inv b\in\frakp\inv$. Hence $\frakp\inv\neq\calO$, since $a\inv b\notin\calO$.

	Now let $\fraka\neq 0$ be an ideal of $\calO$ and $\alpha_1,\dots,\alpha_n$ a system of generators for $\fraka$. We assume that $\fraka\frakp\inv=\fraka$ for the sake of contradiction. Then for every $x\in\frakp\inv$ we have $x\alpha_i\in\fraka$, so
	\[
		x\alpha_i = \sum_j a_{ij}\alpha_j.
	\]
	Writing $A=(x\delta_{ij} - a_{ij})$ we get $A(\alpha_1,\dots,\alpha_n)^t=0$, so with $d=\det(A)$ we have $d\alpha_1=\cdots=d\alpha_n=0$, which implies that $d=0$. Hence $x$ solves the monic polyonomial $f(X)=\det(X\delta_{ij}-a_{ij})$ with coefficients in $\calO$, i.e. $x$ is integral so $x\in\calO$. The choice of $x$ was arbitrary, so $\frakp\inv=\calO$, a contradiction.
	
\end{proof}

\subsection{Unique factorization}

We are now ready to tackle unique factorization in Dedekind domains: we will do so in two steps.

\emph{Existence.} Let $\frakM$ be the set of ideals in $\calO$ which do not admit a unique factorization in prime ideals, excluding $(0)$ and $(1)$, and assume it is non-empty. As before, order $\frakM$ by inclusion and choose a maximal element $\fraka$. It is a basic result that $\fraka$ is contained in a maximal (i.e. prime) ideal $\frakp$, and the trivial inclusion $\calO\subseteq\frakp\inv$ gives $\fraka\subseteq\fraka\frakp\inv$, which together with $\fraka\subseteq\frakp$ give
\[
	\fraka\subseteq\fraka\frakp\inv\subseteq\frakp\frakp\inv\subseteq\calO.
\]

We also have $\frakp\subseteq\frakp\frakp\inv$, and our last lemma gives \emph{proper} inclusions $\frakp\subset\frakp\frakp\inv$ and $\fraka\subset\fraka\frakp\inv$, but $\frakp$ is maximal, so necessarily $\frakp\frakp\inv=\calO$.

Looking back, we have that $\fraka$ is maximal in $\frakM$, and proper inclusions $\fraka\subset\fraka\frakp\inv\subset\calO$. Since $\fraka\frakp\inv$ admits a factorization in prime ideals $\fraka\frakp\inv=\frakp_1\cdots\frakp_r$, so does $\fraka$ by $\fraka=\frakp_1\cdots\frakp_r\frakp$, which is a contradiction. Hence $\frakM$ is empty, and every ideal of $\calO$ can be factorized as a product of prime ideals.

\emph{Uniqueness.} Let
\[
	\fraka = \frakp_1\cdots\frakp_r = \frakq_1\cdots\frakq_s
\]
be two factorizations of $\fraka$ in prime ideals. Then $\frakp_1\mid\frakq_i$ for some $i$, say $\frakq_1$ (order is not important since we work in commutative rings). But $\frakp$ being maximal implies $\frakp_1=\frakq_1$. Cancel $\frakp_1$ on both sides, and we get
\[
	\frakp_2\cdots\frakp_r = \frakq_2\cdots\frakq_s,
\]
so repeating this same procedure we get that $r=s$ and that $\frakp_i=\frakq_i$ after possibly reordering, for all $i$. Hence factorization in prime ideals is unique.

We can group factors of the same prime ideal in the factorization of any ideal $\fraka\neq 0$, giving the familiar representation
\[
	\fraka = \frakp_1^{v_1}\cdots\frakp_r^{v_r}
\]
with $v_i>0$.

Abuse of notation is common: if $\fraka=(a)$, we usually write $a$ in place of $(a)$ or $\fraka$.

The following is a proposition that confirms our intuition of divisibility:

\begin{prop}
	If $\fraka\subseteq\frakp$ where $\fraka$ is any ideal and $\frakp$ is prime, then $\frakp$ appears in the factorization of $\fraka$ into prime ideals.
\end{prop}
\begin{proof}
	Write $\fraka=\frakp_1\cdots\frakp_r$ by unique factorization. Then $\frakp_1\cdots\frakp_r\subseteq\frakp$, so $\frakp\mid\frakp_1\cdots\frakp_r$. By $\frakp$ being prime, necessarily $\frakp\mid\frakp_i$ for some $i$, but prime ideals are maximal and thus $\frakp=\frakp_i$.
\end{proof}

\subsection{Chinese remainder theorem}

An analogue of the classical Chinese remainder theorem holds for any commutative ring with identity $\calO$. Consider any ideal $\fraka$ that can be written as a product of pair-wise coprime ideals $\fraka=\fraka_1\cdots\fraka_n$. First of all, we claim that
\[
	\fraka = \bigcap_{i=1}^n \fraka_i.
\]

It is clear that if $\fraka_i\mid\fraka$ for $i=1,\dots,n$ then $\fraka\subseteq\bigcap_{i=1}^n \fraka_i$ by definition of divisibility of ideals. The other direction is not hard: fix $a\in\bigcap_i\fraka_i$, which implies that $a\in\fraka_i$ for all $i$, i.e. $\fraka_i\mid a$. By the $\fraka_i$ being pair-wise coprime, we get that $\fraka_1\cdots\fraka_n\mid a$, i.e. $a\in\fraka$. We are now ready to state the theorem.

\begin{thm}[Chinese remainder theorem]
	Let $\fraka_1,\dots,\fraka_n$ be pair-wise coprime ideals of $\calO$, i.e. $\fraka_i+\fraka_j=\calO$ for $i\neq j$. Then, with $\fraka=\bigcap_{i=1}^n\fraka_i$, we have
	\[
		\calO/\fraka \cong \bigoplus_{i=1}^n\calO/\fraka_i.
	\]
\end{thm}
\begin{proof}
	We have the canonical ring homomorphism $\calO\to\bigoplus_i\calO/\fraka_i$ given by
	\begin{align*}
		a \longmapsto \bigoplus_i a\Mod{\fraka_i}.
	\end{align*}

	Note that $a\in\calO$ is in the kernel if $a\equiv 0\Mod{\fraka_i}$ for all $i$, i.e. $a\in\bigcap_i\fraka_i$. Reverse inclusion is clear, so the kernel of this homomorphism is exactly $\fraka$. It sufficies to show that the map is surjective to apply the first isomorphism theorem, which follows by an induction argument.
\end{proof}



\subsection{Ideal class group}

We return to a Dedekind domain $\calO$. Just like non-zero elements in $\calO$ have fractional inverses in $K$, we define fractional ideals living in $K$ for non-zero ideals in $\calO$.

\begin{defn}
	A \emph{fractional ideal} of $K$ is a finitely generated $\calO$-submodule $\fraka\neq 0$ of $K$.
\end{defn}

To illustrate why this definition captures the notion of a fractional ideal $\fraka\neq 0$ of $K$ being a `fraction', consider the following argument. By the classification theorem of finitely generated $\calO$-submodules in $K$ of our last section, we have that $\fraka$ is a free $\bbZ$-module, i.e. $\fraka=a_1\bbZ\oplus\cdots\oplus a_n\bbZ$, where $a_i\in K$. Let $c$ be the product of each denominator of $a_i$, so $c\in\calO$ and $c\neq 0$. Then $c\fraka = ca_1\bbZ\oplus\cdots\oplus ca_n\bbZ$ is a free $\bbZ$-module with basis in $\calO$, hence an ideal in $\calO$. On the other hand, if $\fraka\neq 0$ is an $\calO$-submodule of $K$ for which there is some $c\in\calO$ with $c\fraka$ an ideal of $\calO$, we again write $c\fraka$ as a free $\bbZ$-module and divide over $c$ to conclude that $\fraka$ is itself a finitely generated $\calO$-submodule in $K$, i.e. a fractional ideal. This is the technical explanation to the notion that $c$ `clears the denominator of $\fraka$'. We call ideals in $\calO$ \emph{integral ideals}, to distinguish them from fractional ideals. Multiplication of fractional ideals is the same as multiplication of their integral counterparts.


\begin{prop}
	The fractional ideals of a Dedekind domain $\calO$ in its field of fractions $K$ form an abelian group, called the \emph{ideal group} $J_K$ of $K$. The identity is $(1)=\calO$ and the inverse of an integral ideal $\fraka$ is $\fraka\inv$.
\end{prop}
\begin{proof}
	A product of modules is again a module, and in particular the product of two finitely generated $\calO$-submodule is again a finitely generated $\calO$-submodule. Associativity and commutativity are clear. Note that $\fraka\calO\subset\fraka$ by definition of $\calO$-submodule, and for any $x\in\fraka$ we clearly have $x=x\cdot 1$ (identity of $\calO$), so $\fraka\subset\fraka\calO$. Hence $\fraka=\fraka\calO$, i.e. $\calO$ is the identity of the ideal group $J_K$.

	We now show that any fractional ideal has a fractional ideal as its inverse. To begin with, let $\frakp$ be a prime integral ideal. We derived in the proof of existence of unique factorization that $\frakp\frakp\inv=\calO$, hence $\frakp\inv$ is the inverse of $\frakp$.

	Now let $\fraka=\frakp_1\cdots\frakp_r$ be an integral ideal. Let $\frakb=\frakp_1\inv\cdots\frakp_r\inv$, so $\fraka\frakb=\calO$ shows that $\frakb$ is an inverse of $\fraka$. We furthermore want to show that $\fraka\inv=\frakb$; note that $\frakb=\fraka\inv\calO$, i.e. $\frakb\subseteq\fraka\inv$. Now take $x\in\fraka\inv$, so $x\fraka\subseteq\calO$ and thus $x\fraka\frakb\subseteq\frakb\calO=\frakb$, but $\fraka\frakb=\calO$ so $x\in\frakb$. Therefore $\fraka\inv=\frakb$.

	Finally let $\fraka$ be a fractional ideal. We have some non-zero $c\in\calO$ for which $c\fraka\subseteq\calO$ is an integral ideal, which has $(c\fraka)\inv = \{x\in K\mid x(c\fraka)\subseteq\calO\} = \{x\in K\mid (xc\inv)\fraka\subseteq\calO\} = c\inv\fraka\inv$ as its inverse. Then $(c\fraka) (c\inv\fraka\inv)=\calO$ so $\fraka\fraka\inv=\calO$, i.e. $\fraka\inv$ is the inverse of $\fraka$.
\end{proof}

\begin{cor}
	Every fractional ideal $\fraka$ admits a unique representation as a product
	\[
		\fraka = \prod_{\frakp} \frakp^{v_\frakp}
	\]
	with $v_{\frakp}\in\bbZ$ and $v_{\frakp}=0$ for allmost all $\frakp$.
\end{cor}
\begin{proof}
	We first claim that any fractional ideal $\fraka$ can be written as a quotient of two integral ideals. Recall that we have some $c\in\calO$ for which $c\fraka$ is an integral ideal, say $\frakb$. Note that this still holds if we replace $c$ by the principal ideal it generates, i.e. $(c)\fraka=\frakb$. Letting $\frakc=(c)$, we conclude that $\fraka=\frakb\frakc\inv$, so every fractional ideal is indeed of this form.

	Now, integral ideals admit a unique factorization into prime ideals, and in particular $\frakc\inv$ does as well by our previous proof, so $\fraka$ admits a unique factorization into prime ideals.
\end{proof}

This statement of unique factorization is equivalent to saying that $J_K$ is the free abelian group on the set of non-zero prime ideals $\frakp$ of $\calO$.

We denote by $P_K$ the subgroup of fractional \emph{principal} ideals $(a)=a\calO$, $a\in K^*$, and form the \emph{ideal class group} $Cl_K=J_K/P_K$. Note that this group is trivial iff $\calO$ is a principal ideal domain: if it is trivial, then every fractional ideal is principal, and in particular \emph{integral} ideals (ideals of $\calO$) are principal, so $\calO$ is a PID. If $\calO$ is a PID, we get unique factorization into prime ideals which are furthermore principal, and by a similar argument as our last proof we conclude that fractional ideals also factorize uniquely into prime ideals, a fortiori principal, so the ideal class group is trivial.

A good summary of the work we have done so far can be expressed in terms of the following exact sequence:
\[
	1 \to \calO^* \to K^* \to J_K \to Cl_K \to 1,
\]
where the map $\calO^*\to K^*$ is inclusion, $K^*\to J_K$ sends units $a$ to the principal ideal $(a)$ they generate, and $J_K\to Cl_K$ is the canonical quotient homomorphism. Intuitively, if $Cl_K$ turned out to be infinite, it would be as a consequence of having added `too many' ideals to $\calO$ to make unique factorization work. It is remarkable that the \emph{class number} (i.e. the cardinality of $Cl_K$) can be shown to be always finite (but not quite predictable).
