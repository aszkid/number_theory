\section{A primer of algebraic fields}

The classical procedure in number theory is to consider finite extensions $K$ of the field $\bbQ$ by adjoining roots:

\begin{center}
	\begin{tikzcd}
		\calO_L \arrow[r,hook] & L\\
		\calO_K \arrow[u]\arrow[r,hook] & K \arrow[u]\\
		\bbZ \arrow[r,hook]\arrow[u,hook] & \bbQ \arrow[u,hook]
	\end{tikzcd}
\end{center}


Number theorists particularly care about the structure of $\calO_L$ and $\calO_K$, the integral extensions 
\begin{itemize}
	\item What are the units of $\calO_K$ and $\calO_L$?
	\item How do primes $(\mathfrak{p})$ of $\calO_K$ split in $\calO_l$?
\end{itemize}

The first question culminates in \emph{Dirichlet's unit theorem}, which fully describes (???). The second one corresponds to \emph{ramification theory}.
