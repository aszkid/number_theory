\documentclass[psamsfonts]{amsart}

%-------Packages---------
\usepackage[colorinlistoftodos]{todonotes}
\usepackage{mathtools}
\usepackage{amssymb,amsfonts,amsmath,amsthm}
\usepackage{enumerate}
\usepackage{mathrsfs}
\usepackage[english]{babel}
\usepackage{tikz}
\usepackage{tikz-cd}

%--------Theorem Environments--------
%theoremstyle{plain} --- default
\newtheorem{thm}{Theorem}[section]
\newtheorem{cor}[thm]{Corollary}
\newtheorem{prop}[thm]{Proposition}
\newtheorem{lem}[thm]{Lemma}
\newtheorem{conj}[thm]{Conjecture}
\newtheorem{quest}[thm]{Question}

\theoremstyle{definition}
\newtheorem{defn}[thm]{Definition}
\newtheorem{defns}[thm]{Definitions}
\newtheorem{con}[thm]{Construction}
\newtheorem{exmp}[thm]{Example}
\newtheorem{exmps}[thm]{Examples}
\newtheorem{notn}[thm]{Notation}
\newtheorem{notns}[thm]{Notations}
\newtheorem{addm}[thm]{Addendum}
\newtheorem{exer}[thm]{Exercise}

\theoremstyle{remark}
\newtheorem{rem}[thm]{Remark}
\newtheorem{rems}[thm]{Remarks}
\newtheorem{warn}[thm]{Warning}
\newtheorem{sch}[thm]{Scholium}

\makeatletter
\let\c@equation\c@thm
\makeatother
\numberwithin{equation}{section}

\bibliographystyle{plain}

%------- Math definitions -----------
% !TEX root=main.tex

\newtheorem{theorem}{Theorem}[section]
\newtheorem{proposition}[theorem]{Proposition}
\newtheorem{lemma}[theorem]{Lemma}
\newtheorem*{lemma*}{Lemma}
\newtheorem{corollary}[theorem]{Corollary}

\theoremstyle{definition}
\newtheorem{definition}[theorem]{Definition}
\newtheorem*{definition*}{Definition}
\newtheorem{nondefinition}[theorem]{Non-Definition}
\newtheorem{exercise}[theorem]{Exercise}

\newtheorem{axiom}{Axiom}
\newtheorem{remark}[theorem]{Remark}
\newtheorem{warning}[theorem]{Warning}

\newtheorem{question}[theorem]{Question}
\newtheorem{examples}[theorem]{Examples}

\newenvironment{solution}
  {\begin{proof}[Solution]}
  {\end{proof}}

\numberwithin{equation}{subsection}

\DeclareMathOperator{\ext}{ext}
\DeclareMathOperator{\LP}{LP}

\DeclarePairedDelimiter\ceil{\lceil}{\rceil}
\DeclarePairedDelimiter\floor{\lfloor}{\rfloor}
\DeclarePairedDelimiter\parenth{\left(}{\right)}
\DeclarePairedDelimiter\bracket{\langle}{\rangle}

\providecommand{\abs}[1]{\lvert #1 \rvert}

\renewcommand{\_}[1]{\underline{ #1 }}

%% bold math capitals
\newcommand{\bA}{\mathbf{A}}
\newcommand{\bB}{\mathbf{B}}
\newcommand{\bC}{\mathbf{C}}
\newcommand{\bD}{\mathbf{D}}
\newcommand{\bE}{\mathbf{E}}
\newcommand{\bF}{\mathbf{F}}
\newcommand{\bG}{\mathbf{G}}
\newcommand{\bH}{\mathbf{H}}
\newcommand{\bI}{\mathbf{I}}
\newcommand{\bJ}{\mathbf{J}}
\newcommand{\bK}{\mathbf{K}}
\newcommand{\bL}{\mathbf{L}}
\newcommand{\bM}{\mathbf{M}}
\newcommand{\bN}{\mathbf{N}}
\newcommand{\bO}{\mathbf{O}}
\newcommand{\bP}{\mathbf{P}}
\newcommand{\bQ}{\mathbf{Q}}
\newcommand{\bR}{\mathbf{R}}
\newcommand{\bS}{\mathbf{S}}
\newcommand{\bT}{\mathbf{T}}
\newcommand{\bU}{\mathbf{U}}
\newcommand{\bV}{\mathbf{V}}
\newcommand{\bW}{\mathbf{W}}
\newcommand{\bX}{\mathbf{X}}
\newcommand{\bY}{\mathbf{Y}}
\newcommand{\bZ}{\mathbf{Z}}

%% blackboard bold math capitals
\newcommand{\bbA}{\mathbb{A}}
\newcommand{\bbB}{\mathbb{B}}
\newcommand{\bbC}{\mathbb{C}}
\newcommand{\bbD}{\mathbb{D}}
\newcommand{\bbE}{\mathbb{E}}
\newcommand{\bbF}{\mathbb{F}}
\newcommand{\bbG}{\mathbb{G}}
\newcommand{\bbH}{\mathbb{H}}
\newcommand{\bbI}{\mathbb{I}}
\newcommand{\bbJ}{\mathbb{J}}
\newcommand{\bbK}{\mathbb{K}}
\newcommand{\bbL}{\mathbb{L}}
\newcommand{\bbM}{\mathbb{M}}
\newcommand{\bbN}{\mathbb{N}}
\newcommand{\bbO}{\mathbb{O}}
\newcommand{\bbP}{\mathbb{P}}
\newcommand{\bbQ}{\mathbb{Q}}
\newcommand{\bbR}{\mathbb{R}}
\newcommand{\bbS}{\mathbb{S}}
\newcommand{\bbT}{\mathbb{T}}
\newcommand{\bbU}{\mathbb{U}}
\newcommand{\bbV}{\mathbb{V}}
\newcommand{\bbW}{\mathbb{W}}
\newcommand{\bbX}{\mathbb{X}}
\newcommand{\bbY}{\mathbb{Y}}
\newcommand{\bbZ}{\mathbb{Z}}

%% script math capitals
\newcommand{\sA}{\mathscr{A}}
\newcommand{\sB}{\mathscr{B}}
\newcommand{\sC}{\mathscr{C}}
\newcommand{\sD}{\mathscr{D}}
\newcommand{\sE}{\mathscr{E}}
\newcommand{\sF}{\mathscr{F}}
\newcommand{\sG}{\mathscr{G}}
\newcommand{\sH}{\mathscr{H}}
\newcommand{\sI}{\mathscr{I}}
\newcommand{\sJ}{\mathscr{J}}
\newcommand{\sK}{\mathscr{K}}
\newcommand{\sL}{\mathscr{L}}
\newcommand{\sM}{\mathscr{M}}
\newcommand{\sN}{\mathscr{N}}
\newcommand{\sO}{\mathscr{O}}
\newcommand{\sP}{\mathscr{P}}
\newcommand{\sQ}{\mathscr{Q}}
\newcommand{\sR}{\mathscr{R}}
\newcommand{\sS}{\mathscr{S}}
\newcommand{\sT}{\mathscr{T}}
\newcommand{\sU}{\mathscr{U}}
\newcommand{\sV}{\mathscr{V}}
\newcommand{\sW}{\mathscr{W}}
\newcommand{\sX}{\mathscr{X}}
\newcommand{\sY}{\mathscr{Y}}
\newcommand{\sZ}{\mathscr{Z}}


\newcommand{\calG}{\mathcal{G}}
\newcommand{\calH}{\mathcal{H}}
\newcommand{\calR}{\mathcal{R}}
\newcommand{\calF}{\mathcal{F}}
\newcommand{\calD}{\mathcal{D}}
\newcommand{\calC}{\mathcal{C}}
\newcommand{\calA}{\mathcal{A}}
\newcommand{\calQ}{\mathcal{Q}}
\newcommand{\calE}{\mathcal{E}}
\newcommand{\calM}{\mathcal{M}}
\newcommand{\calL}{\mathcal{L}}
\newcommand{\calT}{\mathcal{T}}
\newcommand{\calB}{\mathcal{B}}
\newcommand{\calP}{\mathcal{P}}
\newcommand{\calO}{\mathcal{O}}


\providecommand{\ar}{\rightarrow}
\providecommand{\arr}{\longrightarrow}
\providecommand{\sarr}{\rightarrow}

\providecommand{\sm}{\setminus}

\newcommand{\inv}{^{-1}}

\newcommand{\RHS}{\text{RHS}}
\newcommand{\LHS}{\text{LHS}}

\newcommand{\hint}[2][Hint]{

	(#1: #2)
}


\newcommand{\vx}{\mathbf{x}}
\newcommand{\vy}{\mathbf{y}}
\newcommand{\vz}{\mathbf{z}}
\newcommand{\va}{\mathbf{a}}
\newcommand{\vb}{\mathbf{b}}
\newcommand{\vp}{\mathbf{p}}
\newcommand{\ve}{\mathbf{e}}
\newcommand{\vh}{\mathbf{h}}


\newcommand{\frakS}{\mathfrak{S}}
\newcommand{\fraka}{\mathfrak{a}}
\newcommand{\frakc}{\mathfrak{c}}
\newcommand{\frakp}{\mathfrak{p}}
\newcommand{\frakq}{\mathfrak{q}}
\newcommand{\frakb}{\mathfrak{b}}
\newcommand{\frakM}{\mathfrak{M}}
\newcommand{\frakN}{\mathfrak{N}}


% vector stuff
\DeclarePairedDelimiterX{\inp}[2]{\langle}{\rangle}{#1, #2}
\DeclarePairedDelimiterX{\norm}[1]{\lVert}{\rVert}{#1}



\newcommand{\bdry}{\text{bdry}}
\newcommand{\rng}{\text{range}}
\newcommand{\diam}{\text{diam}}

\newcommand{\seq}[1]{\{#1\}}

% analysis stuff
\DeclareMathOperator{\di}{d\!}

\newcommand{\supp}{\text{supp}}
\newcommand{\Vol}{\text{Vol}}
\newcommand{\Int}{\text{Int}}
\newcommand{\Hom}{\text{Hom}}


% algebra stuff
\DeclareMathOperator{\im}{Im}
\DeclareMathOperator{\Tr}{Tr}
\DeclareMathOperator{\N}{N}
\DeclareMathOperator{\adj}{adj}
\DeclareMathOperator{\rank}{rank}
\DeclareMathOperator{\vol}{vol}



\newcommand{\Mod}[1]{\ (\mathrm{mod}\ #1)}


%--------Meta Data: Fill in your info------
\title{Number Fields: An Introduction to Algebraic Number Theory}

\author{Pol Gómez Riquelme}

\date{Summer of 2018, June-August.}

\begin{document}

\begin{abstract}

This paper is a first introduction to number theory from a modern perspective, through algebraic methods. We review the necessary algebra to understand later material, present the usual setup of number fields, and develop the basic theory of unique factorization of Dedekind domains, prove the finiteness of the class number, and study the structure of the group of units of a given number field.

\end{abstract}

\maketitle

\tableofcontents

\section*{Introduction}

This paper is an attempt at a first introduction to number theory through a somewhat modern lens. With algebra as the fundamental language through which number theory is studied, the ``contemporary'' approach (insofar as ``classical'' number theory consists of essentially ad-hoc methods with little or no underlying framework) can seem daunting because of the machinery required: we aim at providing a survey of the first seven chapters of Jürgen Neukirch's popular \emph{Algebraic Number Theory}, with added background and slightly more prosaic motivations to fundamental concepts, the importance of which can only be fully grasped in retrospective.

The first chapter is a quick review of basic vocabulary in algebra. The second introduces the premise of algebraic number theory: a setup in terms of field theory, with heavy usage of the language of modules. The chapter contains fundamental discussion on integrality, and the classification theorem of finitely generated modules over the integral closure of some number field extension. While undeniably a mouthful, it is our hope that this paper will provide some intuition and the background necessary to make sense of the previous statement. The third chapter builds up towards two major results: it proves unique factorization in Dedekind domains (the natural setup of algebraic number theory), and presents the notion of the class group (which measures the ``expansion'' that has taken place by passing to Dedekind domains). The fourth chapter develops the theory of Minkowski to prove that the class group has finite cardinality (thus showing that this ``expansion'' is somehow controlled). Finally, the fifth and last chapter concludes with a structure theorem of the unit group of a number field (Dirichlet's unit theorem) -- and an application to the solutions of Pell's equation.

While necessarily limited in scope, we nevertheless hope that this paper provides an accessible survey to basic (and powerful) results in algebraic number theory.

\section{Algebra fundamentals}

This first section develops the basic theory of rings of integers and the elementary algebraic tools needed to build them. We follow Dummit \& Foote for the most part. The next section will motivate the structures developed here.

\begin{defn}[D\&F pp. 224]
	A ring $R$ with identity $1\neq 0$ is called a \emph{division ring} if every nonzero $a\in R$ has a multiplicative inverse.

	If $R$ is a commutative division ring, we call it a \emph{field}.
\end{defn}

\begin{defn}
	\begin{enumerate}
		\item A nonzero element $a\in R$ is called a \emph{zero divisor} if there is a nonzero $b\in R$ such that $ab=0$.
		\item Assume $R$ has an identity $1\neq 0$. An element $u\in R$ is called a \emph{unit} if there is some $v\in R$ such that $uv=vu=1$. The set of units is denoted by $R^\times$, and forms a subgroup under multiplication (the \emph{group of units}).
	\end{enumerate}
\end{defn}

Fields are rings for which every nonzero element is a unit. Units cannot be zero divisors, for if $a\in R$ is such an element, we have $c,d\in R$ with $c\neq 0$ such that $ac=0$ and $ad=da=1$, but $c=1c=(da)c=d(ac)=d0=0$, a contradiction.



\subsection{Valuation rings}

\begin{defn}
	Given a field $K$, a \emph{discrete valuation} on $K$ is a function $v:K^\times\to\bbZ$ satisfying
	\begin{enumerate}[(i)]
		\item $v(ab) = v(a)+v(b)$, i.e. it is a group homomorphism;
		\item $v$ is surjective; and
		\item $v(a+b) = \min\{v(a), v(b)\}$ for all $a,b\in K^\times$ with $a+b\neq 0$.
	\end{enumerate}

	The set $R = \{x\in K^\times \mid v(x)\geq 0\}\cup\{0\}$ is called the \emph{valuation ring} of $v$.
\end{defn}

First, note that $v(1)=0$, since for all $x\in K^\times$, $v(x)=v(1x)=v(1)+v(x)$. We show that $x\in R$ is a unit in $R$ iff $v(x)=0$: if $x$ is a unit in $R$, we have some $y\in R$ such that $xy=yx=1$. Hence $0=v(1)=v(xy)=v(x)+v(y)$: both quantities are positive by $x,y\in R$, so necessarily $v(x)=v(y)=0$. Now fix $x\in R$: note that $v(x)$ is not well-defined if $x=0$, so we may assume $x\neq 0$, and thus necessarely $x\in K^\times$, with $v(x)=0$. In particular, we have some $y\in K^\times$ such that $xy=yx=1$. But notice that $0=v(1)=v(xy)=v(x)+v(y)=v(y)$, so $y\in R$ and therefore $x$ is a unit in $R$.

To illustrate the notion of a valuation, we note the following example (D\&F, Exercise 7.1.27): let $p$ be a prime number, $K=\bbQ$, and define $v_p:\bbQ^\times\to\bbZ$ by $v_p(a/b)=\alpha$ where $a/b=p^\alpha c/d$ and $p\nmid c$, $p\nmid d$. We prove that the corresponding valuation ring of $v_p$ is the ring of all rational numbers whose denominators are relatively prime to $p$, and we describe the units.

We have $R=\{a/b\in\bbQ^\times \mid v_p(a/b)\geq 0\}$, i.e. $R$ is the set of all rationals $a/b$ such that $a/b=p^\alpha c/d$ for $c,d\in\bbZ$ chosen adequately and $\alpha\geq 0$. In particular, this implies that there is no $p$ in the denominator of the expression, so $p\nmid b$, i.e. $(p,b)=1$. On the other hand, if $p\nmid b$, we let $\alpha$ be the largest non-negative integer such that $p^\alpha \mid a$. Then, letting $c=a/p^\alpha$, we have $a/b=p^\alpha c/b$. Hence $a/b\in R$.

As for the units, by our previous observation we have that $a/b\in R$ is a unit iff $v_p(a/b)=0$, i.e. $a/b=p^0 c/d$, forcing $c=a$, $d=b$. Hence the units of $R$ are all rationals $a/b$ with numerator \emph{and} denominator $p$-free.




\subsection{Polynomial rings}

For any commutative ring $R$ with identity, we define the ring of polynomials in the variable $x$ over $R$, denoted by $R[x]$, as all formal sums of the form
\[
	a_nx^n+a_{n-1}x^{n-1}+\cdots + a_1x + a_0,
\]
where $x$ is an `indeterminate', $a_nx^n$ the leading term, $a_n$ the leading coefficient, and $n$ the degree of the polynomial. A polynomial is monic if $a_n=1$.

We define addition of polynomials component-wise, and multiplication by distributivity defining $(ax^i)(bx^j)=abx^{i+j}$; this makes $R[x]$ into a ring

$R$ can be naturally embedded into $R[x]$ as the constant polynomials; this gives $R[x]$ an identity, and $R$ being commutative makes $R[x]$ commutative.



\subsection{Ring homomorphisms and ideals}

Ideals naturally arise as kernels of ring homomorphisms, just like normal subgroups spring from group homomorphisms.

\begin{defn}
	A \emph{ring homomorphism} is a map $\phi:R\to S$ of rings satisfying
	\begin{enumerate}[(i)]
		\item $\phi(a+b)=\phi(a)+\phi(b)$ and
		\item $\phi(ab)=\phi(a)\phi(b)$
	\end{enumerate}
	for all $a,b\in R$.

	The \emph{kernel} of $\phi$ is defined by $\ker\phi=\phi\inv(0)$.
\end{defn}

\begin{prop}
	For any ring homomorphsim $\phi:R\to S$, its image and kernel are subrings of $S$ and $R$ respectively.
\end{prop}

In fact, the kernel has a somewhat stronger property: for any $r\in R$ and $k\in\ker\phi$, we have that $\phi(rk)=\phi(r)\phi(k)=\phi(r)0=0$, so $rk,kr\in\ker\phi$. This is a normality property; in ring theoretic language, we say that $\ker\phi$ is an \emph{ideal} of $R$.

\begin{defn}
	Let $I\subset R$ be a subring. Define $rI=\{ra \mid a\in I\}$ and $Ir=\{ar \mid a\in I\}$ for any $r\in R$.

	We say that $I$ is an \emph{ideal} if we always have the inclusions $rI\subset I$ and $Ir\subset I$.
\end{defn}

(There is a general distinction between \emph{left-} and \emph{right-}ideals, but they agree for commutative rings. In particular, number fields and integral rings are commutative, so we are content with sticking to the commutative case.)

Just like we do for groups, we can partition $R$ in cosets $r+I$, and endow the set of cosets with the ring structure inherited from $R$: namely, defining $(r+I)+(s+I)=(r+s)+I$ and $(r+I)\cdot (s+I) = (r\cdot s) + I$. And again, this is only well-defined if $I$ is an ideal, not just a subring of $R$. The set of cosets under this ring structure is denoted by $R/I$ and called the \emph{quotient ring} of $R$ modulo $I$.

The group isomorphism theorems carry nicely to rings: every ideal of a ring is indeed the kernel of some ring homomorphism, etc.

\emph{Example;} consider $\bbZ[x]$, and let $I$ be the set of polynomials with degree $\geq 2$. It is clear that the product of any $p(x)\in\bbZ[x]$ with such a polynomial is at least of degree 2. Hence $I$ is an ideal, and we can form the quotient ring $\bbZ[x]/I$. Two polynomials $p(x),q(x)\in\bbZ[x]$ are identified in the quotient if $p(x)-q(x)=r(x)$ for some $r(x)\in I$; i.e. they differ by a degree $\geq 2$ polynomial, which is only the case if they agree on their degree 0 and 1 terms.

Hence $\overline{(1+3x)}+\overline{(-4+5x)}=\overline{-3+8x}$ and $\overline{(1+3x)}\overline{(-4+5x)}=\overline{-4+5x-12x+15x^2}=\overline{-4-7x}$.

Within the richer ring structure, we can define operations between ideals:

\begin{defn}
	Let $I$ and $J$ be ideals of $R$.
	\begin{enumerate}[(1)]
		\item Their sum is given by $I+J=\{a+b \mid a\in I, b\in J\}$.
		\item Their product is denoted by $IJ$ and defined as the set of all finite sums of pairs $ab$ with $a\in I$, $b\in J$.
	\end{enumerate}
\end{defn}

We restrict our attention to rings with identity $1\neq 0$.

\begin{defn}
	Given any subset $A$ of a ring $R$, we denote by $(A)$ the smallest ideal of $R$ containing $A$, and call it the \emph{ideal generated by $A$}.

	An ideal generated by a single element is called a \emph{principal ideal}.

	An ideal generated by a finite set is called a \emph{finitely generated ideal}.
\end{defn}

We can precisely obtain $(A)$ by taking all ideals $I$ such that $A\subset I$; since at least $R\subset I$, their intersection is not empty, so $(A)=\bigcap_{A\subset I} I$.

We also note that ideals truly capture in some sense the arithmetic notion of divisibility, familiar from $\bbZ$: an element $b\in R$ is in the ideal $(a)$ generated by some other $a\in R$ iff $b=ar$ for some $r\in R$, i.e. if $b$ is a \emph{multiple} of $a$, or equivalently if $a$ divides $b$. Furthermore, $b\in(a)$ iff $(b)\subset(a)$.

\begin{defn}
	An ideal $M\subset S$ of a ring $S$ is \emph{maximal} if $M\neq S$ and the only ideals that contain $M$ are $M$ itself and the entire ring $S$.
\end{defn}

\begin{prop}
	For any commutative ring $R$, an ideal $M$ is maximal iff $R/M$ is a field.
\end{prop}




\subsection{Fields and embeddings}









\section{Algebraic number fields}

It is not a stretch to say that number theory, at least in the classical sense, is about numbers. The study of solutions to equations like $x^2+y^2=z^2$ with ``integer'' coefficients (denoted by $\bbZ$) have concerned number theorists since the Babylonians~\cite{rouseball}. Passing over to the ``rational'' numbers (denoted by $\bbQ$), as ratios of integer parts, is only natural in working with integer equations. Starting with Pythagoras, the Greeks decidedly only worked with rational numbers, and treated ``irrational'' magnitudes as non-existent, as non-numbers (\emph{``He is unworthy of the name of man who is ignorant of the fact that the diagonal of a square is incommesurable with its side''} -- Plato, as quoted by Sophie Germain). While number theory nowadays is still primarily concerned with ``integral'' quantities, mathematics has thankfully moved forward, and the tools of complex analysis and the entire repertoire of numbers (whether real or ``imaginary'') is available to those studying number theoretic questions.

The basic operations of arithmetic in $\bbZ$ (i.e. one can add, subtract, and multiply) are captured in the more general notion of a \emph{ring}. As early as in the 16th century, mathematicians found ``imaginary'' as useful as metaphysically perplexing. One can consider numbers of the form $a+bi$ where $i=\sqrt{-1}$ is the ``imaginary unit'', and show that they not only form a ring; they form an Euclidean domain, admitting unique factorization in prime elements. Gauss used them extensively \todo{citation!}. In other words, we can ``extended'' the usual integers and still preserve essential properties.

Algebra turns this into a rigurous process. We start with the rational numbers, and consider a finite field extension $F$ over $\bbQ$:
\begin{center}
\begin{tikzcd}
		& F \\
	\bbZ \arrow[r] & \bbQ \arrow[u]
\end{tikzcd}
\end{center}
(arrows denote inclusion for now). Previously, we formed ``fractional'' elements $\bbQ$ from $\bbZ$: now we would like to restrict our attention to ``integral'' elements of $F$, i.e. forming ``$\bbZ$'' from ``$\bbQ$''. If we can define such a notion that works for $\bbQ$, that will suffice. For example, consider polynomials of degree one with coefficients in $\bbZ$, i.e. $x+a=0$ with $a\in\bbZ$. The solutions to these equations in $\bbQ$ are exactly the elements of $\bbZ$: in fact, considering polynomials with higher degree yields no other solutions in $\bbQ$. We say that the \emph{integral closure} of $\bbZ$ in $\bbQ$ is, again, $\bbZ$.

In the general case, we go back to our finite extension $F$ and consider monic polynomials
\[
	x^n + a_1x^{n-1} + \cdots + a_{n-1}x + a_n = 0
\]
with coefficients $a_i\in\bbZ$. We say that an element $y\in F$ is \emph{integral} if it solves such a polynomial, and call the set of integral elements of $F$ the \emph{integral closure} of $\bbZ$ in $F$, denoted by $\calO_F$. The diagram above becomes

\begin{center}
\begin{tikzcd}
	\calO_F \arrow[r] & F\\
	\bbZ \arrow[u] \arrow[r] & \bbQ \arrow[u].
\end{tikzcd}
\end{center}

It requires some work to show that if $F=\bbQ(i)$, then its integral closure is $\calO_F=\bbZ[i]$ (the usual Gaussian integers), so that our definition does make sense.

To what extend $\calO_F$ is a good generalization of the usual integers $\bbZ$ is the guiding thread of this paper. In particular, we will ask

\begin{quest}
	Does unique factorization hold in $\calO_F$?
\end{quest}

and, from a more algebraic point of view,

\begin{quest}
	What are the units of $\calO_F$?
\end{quest}


\subsection{Modules and integrality}

The question of integrality is fundamental in the study of number fields, and it proves fruitful to develop the theory in its full generality.

For a given ring extension $A\subseteq B$, we say that $B$ is \emph{integral over $A$} if each element in $B$ is integral over $A$. Integral elements of $B$ give the $A$-modules they generate a good structure property.

\begin{prop}[Neukirch 2.2]
	Finitely many elements $b_1,\dots,b_n\in B$ are integral over $A$ iff the $A$-module $A[b_1,\dots,b_n]$ is finitely generated.
\end{prop}
\begin{proof}
	($\Longrightarrow$) Start by assuming that $b\in B$ is integral over $A$. Then $f(b)=0$ for some polynomial $f(x)$ over $A$. Now let $g(x)\in A[x]$: by the division algorithm, we have $g(x)=q(x)f(x)+r(x)$ with polynomials over $A$ and $\deg r(x) < \deg q(x)$.

	In particular, passing to $A[b]$, we have $g(b)=r(b)=a_0+a_1b+\cdots+a_{n-1}b^{n-1}$. So any element in $A[b]$ is generated by $\{1,b,\dots,b^{n-1}\}$, i.e. $A[b]$ is finitely generated.

	Now assume that $R=A[b_1,\dots,b_{n-1}]$ is finitely generated for $b_1,\dots,b_n\in B$ integral over $A$. The same argument shows that $R[b_n]$ is finitely generated too, since $b_n$ is integral over $A$ and hence over $R$ as well. Induction proves the result.

	($\Longleftarrow$) Now assume that $A[b_1,\dots,b_n]$ is finitely generated. Let $\omega_1,\dots,\omega_r$ be a system of generators. Take any $b\in A[b_1,\dots,b_n]$; we have $\omega_i A[b_1,\dots,b_n] \subseteq A[b_1,\dots,b_n]$, so in particular $\omega_i b = a_{i1}\omega_1 + \cdots + a_{ir}\omega_r$ for each $1\leq i \leq r$. Rearranging the equations we get a system
	\begin{align*}
		(b-a_{11})\omega_1 + \cdots + (-a_{1r})\omega_r &= 0\\
		\vdots &= 0\\
		(-a_{i1})\omega_1 + \cdots + (b-a_{ii})\omega_i + \cdots + (-a_{ir})\omega_r&=0\\
		\vdots &= 0\\
		(-a_{r1})\omega_1 + \cdots + (b-a_{rr})\omega_r &= 0
	\end{align*}
	In matrix form we can write the coefficients as $bI - (a_{ij})$, and we have that the vector of generators solves the system, i.e. $(bI-(a_{ij}))\omega_i = 0$, so $\det(bI-(a_{ij}))\omega_i = 0$, but $\omega_i\neq 0$ so necessarily $\det(bI-(a_{ij}))=0$. This gives a polynomial over $A$ for which $b$ is a root, so $b$ is integral over $A$.
\end{proof}

Compare this result to an arbitrary ring extension, which is not necessarily finitely generated as an $A$-module, just like the polynomial ring $A[x]$.

\begin{prop}
	Let $A\subseteq B\subseteq C$ be ring extensions. If $B$ is integral over $A$ and $C$ is integral over $B$, then $C$ is integral over $A$.
\end{prop}
\begin{proof}
	Fix $c\in C$: since $C$ is integral over $B$, we have an equation $c^n+b_1c^{n-1}+\cdots+b_n=0$ with coefficients in $B$. In particular $R=A[b_1,\dots,b_n]$ is finitely generated because $B$ is integral over $A$. As an $R$-module, $R[c]$ is finitely generated since $c$ is integral over $B$. Since $B$ is integral over $A$, then $R[c]$ is furthermore finitely generated over $A$ by each $b_i$ being integral over $A$. Hence $c$ is integral over $A$.
\end{proof}

We call the set $\overline{A}=\{b\in B \mid b \text{ integral over } A\}$ the \emph{integral closure of $A$ in $B$}, and say that $A$ is \emph{integrally closed} if $A=\overline{A}$. Note that $A$ is trivially integral over $A$ because $a\in A$ is a solution to the monic polynomial $x-a=0$, hence $A\subset\overline{A}$. This gives a chain $A\subset \overline{A}\subset\overline{\overline{A}}$ of integral closures, and applying the previous proposition we get that $\overline{\overline{A}}$ is integral over $A$, so $\overline{\overline{A}}\subset\overline{A}$. The two inclusions imply that $\overline{A}=\overline{\overline{A}}$, i.e. $\overline{A}$ is integrally closed, so it makes sense to call it \emph{the} integral closure of $A$.


\subsection{Trace and norm}

Consider a finite field extension $L\mid K$. The following concepts will prove useful in studying the integral elements of $L$.\todo{Is it so?}

\begin{defn}
	The \emph{trace} and \emph{norm} of some $x\in L$ are defined as the trace and norm of the matrix $[T_x]$ representing the endomorphism $T_x:L\to L$ given by $T_x(\alpha)=x\alpha$, i.e.
	\begin{align*}
		\Tr_{L\mid K}(x) = \Tr([T_x]) & & \N_{L\mid K}(x) = \det([T_x]).
	\end{align*}
\end{defn}

We only need to compute the action of trace and norm on some basis to get the field trace and norm: for example, with $K=\bbR$ and $L=\bbC$, we have a the usual basis $\{1,i\}$ for $\bbC$, and so for any $z=a+bi\in\bbC$ we get $T_z(1)=a+bi$ and $T_z(i)=-b+ai$. This gives the matrix
\[
	\begin{bmatrix}
		a & -b\\
		b & a
	\end{bmatrix},
\]
the trace and norm of which are $2a$ and $a^2+b^2$.

\begin{prop}
	The trace $\Tr_{L\mid K}$ is a $K$-linear additive homomorphism $L\to K$, and the norm $\N_{L\mid K}$ is a \todo{Also K-linear?}multiplicative homomorphism $L^\times\to K^\times$.
\end{prop}
\begin{proof}
	This amounts to passing to matrix representations as needed.

	First of all, it is clear that $T_{x+y}=T_x+T_y$, and $T_{xy}=T_x\circ T_y$. Now pick a basis for $L\mid K$ and find the matrix representations of $T_x$ and $T_y$. We have that
	\begin{align*}
		\Tr_{L\mid K}(x+y)=\Tr([T_{x+y}])&=\Tr([T_x+T_y])\\
			&=\Tr([T_x])+\Tr([T_y])\\
			&= \Tr_{L\mid K}(x)+\Tr_{L\mid K}(y),
	\end{align*}
	so the field trace is at least additive, just like the usual trace.

	Similarly, we have
	\begin{align*}
		\N_{L\mid K}(xy) = \det([T_{xy}]) &= \det([T_x][T_y])\\
			&= \det([T_x])\det([T_y])\\
			&= \N_{L\mid K}(x)\N_{L\mid K}(y),
	\end{align*}
	so the field norm is multiplicative, just like the usual norm.

	For $K$-linearity of the trace, fix $c\in K$ and $x\in L$: as an endomorphism, we have $T_{cx}(\alpha)=c T_x(\alpha)$, so
	\begin{align*}
		\Tr_{L\mid K}(cx) = \Tr([T_{cx}]) &= \Tr(c[T_x])\\
			&= c\Tr([T_x])\\
			&= c \Tr_{L\mid K}(x).
	\end{align*}

	And additivity extends this to finitely many terms, so the trace is $K$-linear.
\end{proof}




Given $n=[L : K]$, the characteristic polynomial of $T_x$ is
\[
	f_x(t) = \det(tI - [T_x]) = t^n - a_1t^{n-1} + \cdots + (-1)^na_n.
\]
Note that $a_1=\Tr_{L\mid K}(x)$ and $a_n=\N_{L\mid K}(x)$\todo{Not obvious!}, and the degree of $f_x(t)$ is always fixed relative to $L\mid K$.

On the other hand, the minimal polynomial $p_x(t)$ of any $x\in L$ over $K$ has degree $[K(x) : K]$, which depends of course on the choice of $x$.



We now study the close interplay between the characteristic and the minimal polynomial.

\begin{prop}
	Every $x \in L$ is a root of the characteristic polynomial of $T_x$.
\end{prop}
\begin{proof}
	This follows from Cayley-Hamilton: considering $f_x$ as a linear operator, evaluating at the matrix $[T_x]$ is zero by definition, but we can expand it with addition being usual matrix addition, and powers being repeated matrix multiplication. Then
	\begin{align*}
		0 = f_x([T_x]) &= [T_x]^n - a_1[T_x]^{n-1} + \cdots + (-1)^na_n I\\
			&= [T_{x^n}] - a_1[T_{x^{n-1}}] + \cdots + (-1)^na_n I\\
			&= [T_{x^n}] - [T_{a_1x^{n-1}}] + \cdots + (-1)^na_n I\\
			&= [T_{x^n-a_1x^{n-1} + \cdots + (-1)^na_n}]\\
			&= [T_{f_x(x)}],
	\end{align*}
	which implies that $T_{f_x(x)}=0$, i.e. $f_x(x)=0$.
\end{proof}

\begin{prop}
	Fix $x\in L$, let $p_x(t)$ be the minimal polynomial of $x$ and $f_x(t)$ the characteristic polynomial of $T_x$. Then
	\[
		f_x(t) = (p_x(t))^{n/d},
	\]
	where $n=[L:K]$ and $d=[K(x):K]$.
\end{prop}
\begin{proof}
	Fix $x\in L$. The set $\{x,x^2,\dots,x^{d-1}\}$ is a basis of $K(x)\mid K$. Pick a basis $\{\alpha_1,\dots,\alpha_m\}$ of $L\mid K(x)$. The set 
	\[
		\{\alpha_1, \alpha_1 x,\alpha_1 x^2, \dots, \alpha_1 x^{d-1}; \alpha_2, \alpha_2 x, \alpha_2 x^2, \dots, \alpha_2 x^{d-1}; \dots ; \alpha_m, \alpha_m x, \dots, \alpha_m x^{d-1}\}
	\]
	is a basis for $L\mid K$.

	We write $T_x$ as a matrix with respect to this basis. Let $e_{ij}$ be the basis vectors of $L$, i.e. $e_{01} = \alpha_1 = (1,0,\dots,0)$, $e_{11} = x \alpha_1 = (0,1,\dots,0)$, etc. Note that for $i<d-1$ we have
	\begin{align*}
		T_x(e_{ij}) = x e_{ij} &= x x^i \alpha_j\\
			&= x^{i+1} \alpha_j\\
			&= e_{(i+1)j},
	\end{align*}
	and for $i=d-1$
	\begin{align*}
		T_x(e_{(d-1)j}) = x^d \alpha_j,
	\end{align*}
	but since $\{x^i\}_{0\leq i \leq d-1}$ is a basis for $K(x)$ we must have some linear combination 
	\begin{align*}
		c_d + x c_{d-1} + \cdots + x^{d-1} c_1 + x^d = 0,
	\end{align*}
	which is in fact the minimal polynomial of $x$ since $d=[K(x):K]$. Hence $T_x(e_{(d-1)j}) = \alpha_j (-c_d - x c_{d-1} - \cdots - x^{d-1} c_1)$.
	
	We have just shown that $T_x$ gives the same matrix for each subspace generated by $\{e_{ij}\}_{0\leq j\leq m}$:
	\[
	\begin{pmatrix}
		0 & 1 & 0 & \cdots & 0\\
		0 & 1 & 1 & \cdots & 0\\
		\cdots & \cdots & \cdots & \cdots & \cdots\\
		0 & 0 & 0 & \cdots & 1\\
		-c_d & -c_{d-1} & -c_{d-2} & \cdots & -c_1
	\end{pmatrix},
	\]
	so to compute the characteristic polynomial of $[T_x]$ it suffices to compute it for the block above. This can be done by using the permutation formula of the determinant\todo{Show work}, which gives
	\[
		t^d + c_1 t^{d-1} + \cdots + c_{d-1} t + c_d = p_x(t).
	\]

	By $[T_x]$ consisting of $m$ identical blocks with determinant $p_x(t)$, we get $f_x(t)=p_x(t)^m$.

\end{proof}

Note that if $L=K(x)$, we have that $f_x(t)=p_x(t)$.


A special form for field trace and norm can be given if $L\mid K$ is a separable extension. Let $d=[L:K(x)]$ and $m=[K(x):K]$, so $[L:K]=dm$.

\begin{prop}
	Given a separable field extension $L\mid K$ and $\sigma : L\to\overline{K}$ a $K$-embedding of $L$, we have
	\begin{enumerate}[(i)]
		\item $f_x(t)= \prod_\sigma(t-\sigma x)$,
		\item $\Tr_{L\mid K}(x) = \sum_\sigma \sigma x$, and
		\item $\N_{L\mid K}(x) = \prod_\sigma \sigma x$.
	\end{enumerate}
\end{prop}
\begin{proof}
	Since $L\mid K$ is separable, $x\in L$ is separable, so there are exactly $m$ $K$-embedding $\tilde{\sigma}:K(x)\to \overline{K}$. Each of these extends to a $K$-embedding $\sigma: L\to \overline{K}$ in $d$ different ways. Since $[L:K]_s=md$, this accounts for all $K$-embeddings $L\to\overline{K}$. Hence we can split $\Hom_K(L,\overline{K})$ into $m$ equivalence classes by identifying $\sigma \sim \tau$ if $\sigma(x)=\tau(x)$ (corresponding to choice of $x^\sigma$); each of the equivalence classes has $d$ distinct embeddings.

	%(Another argument: $L$ is a finite extension over $K(x)$, so any element in $L$ can be written as a linear combination of $d$ basis elements in $L$ with coefficients in $K(x)$. A $K$-embedding of $L$ into $\overline{K}$ must decide where $x$ goes (exactly $m$ choices, the degree of $p_x(t)$) and where the $d$ elements of the basis of $L$ go.)

	Hence, taking representatives $\sigma_1,\dots,\sigma_m$, we have
	\begin{align*}
		p_x(t) = \prod_{i=1}^m (t- \sigma_i x),
	\end{align*}
	and by the previous proposition,
	\begin{align*}
		f_x(t) &= \left( p_x(t)\right)^d\\
			&= \left(\prod_{i=1}^m (t-\sigma_i x)\right)^d\\
			&= \prod_\sigma (t-\sigma x),
	\end{align*}
	the last equality being valid since any two embeddings in the same equivalence class agree on $x$.

	Recall the block form of $[T_x]$. Summing along the diagonal gives $-c_1$, which by Vieta is equal to the sum of the roots of $p_x(t)$, i.e. $\sum_{i=1}^m \sigma_i x$. Again, the block is repeated $d$ times, so we have
	\begin{align*}
		\Tr_{L\mid K}(x) &= d \sum_{i=1}^m \sigma_i x\\
			&= \sum_\sigma \sigma x.
	\end{align*}

	Computing the determinant of $[T_x]$ by the permutation formula gives
	\begin{align*}
		\N_{L\mid K} &= \sum_\sigma \sigma x.
	\end{align*}

\end{proof}

The trace and norm behave well as we take field extensions:

\begin{cor}
	Given a tower of finite field extensions $K\subset L\subset M$, we have
	\begin{align*}
		\Tr_{L\mid K} \circ \Tr_{M\mid L} = \Tr_{M\mid K}, & & \N_{L\mid K} \circ \N_{M\mid L} = \N_{M\mid K}.
	\end{align*}
\end{cor}
\begin{proof}
	Using a similar technique as in the proposition above, we partition $\Hom_K(M,\overline{K})$ into $m=[L:K]$ equivalence classes by identifying $\sigma\sim\tau$ if $\sigma|_L = \tau|_L$. Take a system of representatives $\sigma_1,\dots,\sigma_m$: then $\Hom_K(L,\overline{K}) = \{\sigma_i|_L , 1\leq i \leq m\}$.

	Fix $x\in M$. We have
	\begin{align*}
		\Tr_{M\mid K}(x) &= \sum_\sigma \sigma x\\
			&= \sum_{i=1}^m \sum_{\sigma\sim\sigma_i} \sigma x\\
			&= \sum_{i=1}^m \Tr_{\sigma_i M\mid \sigma_i L}(\sigma_i x)\\
			&= \sum_{i=1}^m \sigma_i \Tr_{M\mid L}(x)\\
			&= \Tr_{L\mid K}(\Tr_{M\mid L}(x)).
	\end{align*}
	\todo{Develop third equality.}

\end{proof}


Given a basis $\alpha_1,\dots,\alpha_n$ of a separable field extension $L\mid K$, we define its \emph{discriminant} by
\[
	d(\alpha_1,\dots,\alpha_n) = \det((\sigma_i\alpha_j))^2,
\]
where $\sigma_i$ varies over $K$-embeddings $L\to\overline{K}$.\todo{Write discriminant in terms of trace and in the special case of a simple extension.}


\begin{prop}
	If $L\mid K$ is separable and $\alpha_1,\dots,\alpha_n$ is a basis for $L$, we have that $d(\alpha_1,\dots,\alpha_n)\neq 0$ and $(x,y) \to \Tr_{L\mid K}(xy)$ is a nondegenerate bilinear form on $L$ as a $K$-vector space.
\end{prop}
\begin{proof}
	A routine computation shows that $(x,y)$ is indeed linear on each coordinate.

	Recall that a bilinear form $(x,y)$ can be expressed by a matrix $M$, so that $(x,y) = x^T M y$. Non-degeneracy is equivalent to requiring $\det(M)\neq 0$.
	
	Since $L\mid K$ is separable, the primitive element theorem gives some $\theta\in L$ for which $L=K(\theta)$. Hence
	\[
		\det(M) = \prod_{i<j}(\theta_i-\theta_j)^2 \neq 0,
	\]
	since otherwise $\theta_i = \theta_j$, implying that $\sigma_i=\sigma_j$, a contradiction.

	Denote the bilinear form $(x,y)$ with respect to some other basis $\alpha_1,\dots,\alpha_n$ by $M'$. Using a change of basis, we have $M=SM'S\inv$. Take the determinant on both sides to get $0\neq\det(M)=\det(S)\det(M')\det(S\inv)=\det(M')$.
\end{proof}



\subsection{Integrality continued -- integral bases}


The main use of integrality we care about in this presentation concerns the ring of integers $\calO_K$ of some algebraic number field $K$. 

We remind ourselves of the usual setup: we have an integrally closed integral domain $A$, its field of fractions $K$, a finite separable extension $L\mid K$, and the integral closure $B$ of $A$ in $L$. The classical equivalent is given by $A=\bbZ$, $K=\bbQ$, $L$ a quadratic or cyclotomic extension, etc., and $B=\calO_K$ the ring of integers.

Note that $x\in B$ implies that all its conjugates $\sigma x$ are integral in $L$ too. Hence $\Tr_{L\mid K}(x)$ and $\N_{L\mid K}(x)$ are in $A$\todo{Elaborate. Use $A=B\cap K$.}.

\begin{lem}
	If $\alpha_1,\dots,\alpha_n$ is a basis of $L\mid K$ contained in $B$, of discriminant $d=d(\alpha_1,\dots,\alpha_n)$, we have
	\[
		dB\subseteq A\alpha_1 + \cdots + A\alpha_n.
	\]
\end{lem}
\begin{proof}
	Fix $\alpha\in B$. We have $\alpha=a_1\alpha_1+\cdots+a_n\alpha_n$ with $a_j\in K$. Taking the trace of both sides of the equality multiplied by $\alpha_i$, we get
	\begin{align*}
		\Tr_{L\mid K}(\alpha_i \alpha) &= \Tr_{L\mid K}(a_1\alpha_1+\cdots+a_n\alpha_n)\\
			&= \sum_{j=1}^n \Tr_{L\mid K}(\alpha_i\alpha_j) a_j,
	\end{align*}
	this last equality by $\Tr$ being additive and $K$-linear. This gives a system of $n$ linear equations for which the $a_j$ are a solution. In fact, since the trace evaluated at elements in $B$ lands in $A$, this is a system with coefficients in $A$; say $Mx=b$.

	The determinant of $M=(\Tr_{L\mid K}(\alpha_i\alpha_j))$ is exactly $d$, so we have
	\begin{align*}
		x &= M\inv b\\
		  &= \det(M)\inv\adj(M)b.
	\end{align*}
	
	Hence we can write $a_j$ as a quotient of some element in $A$ by $d$, i.e. $da_j\in A$. Therefore $d\alpha\in A\alpha_1+\cdots+A\alpha_n$.
\end{proof}

This setup, a system $\omega_1,\dots,\omega_n\in B$ that serves as a basis for $B$ over $A$ proves useful: we say that $\{\omega_j\}$ forms an \emph{integral basis} of $B$ over $A$. Note that such a basis immediately gives a basis of $L\mid K$ (for $L$ is the field of fractions of $B$\todo{Elaborate. Easy argument.}), which forces our integral basis to be of length $[L:K]$. In particular, if $B$ admits an integral basis, we can consider $B$ as a free $A$-module or rank $[L:K]$.

For most of our upcoming work, we require $A$ to be a principal ideal domain. This nicely classifies $B$-submodules of $L$.

\begin{prop}
	If $L\mid K$ is separable and $A$ is a PID, then every finitely generated $B$-submodule $M\neq 0$ is a free $A$-module of rank $[L:K]$.
\end{prop}
\begin{proof}
	Let $M\neq 0$ be a finitely generated $B$-submodule of $L$ and $\alpha_1,\dots,\alpha_n$ a basis of $L\mid K$. We want to apply the previous theorem, but first need to somehow shift the basis to lie in $B$. Fix $\alpha_i$; we have some monic polynomial $p(x)$ with coefficients $a_j$ in $K$ for which $\alpha_i$ is a root. Let $l$ be the product of all the denominators of the $a_j$, so that
	\begin{align*}
		l^n p(\alpha) = l^n \alpha_n + l^n a_1 \alpha^{n-1} + \cdots + l^n a_n = 0.
	\end{align*}

	Now consider the monic polynomial
	\begin{align*}
		q(x) = x^n + l a_1 x^{n-1} + \cdots + l^n a_n,
	\end{align*}
	which has coefficients in $A$ (denominators cancel thanks to $l$). This polynomial has $l\alpha$ as a root, which implies that $l\alpha$ is integral over $A$, that is $l\alpha\in B$. We repeat this procedure for each element of the basis, and hereafter refer to this new basis simply as $\alpha_j\in B$.

	Hence, with $d=d(\alpha_1,\dots,\alpha_n)$, we have $dB\subseteq A\alpha_1+\cdots+A\alpha_n$. In the language of modules, we have that $dB$ is a submodule of the free module $A_1\alpha_1\oplus\cdots\oplus A_n\alpha_n$. In particular, $A$ is a PID, so it is noetherian, and therefore $dB$ is a torsion-free finitely generated module. Since $dB$ has no torsion part, by the classification theorem of finitely generated modules over PIDs it is necessarily a free module of rank $m \leq n = [L:K]$.

	Now notice that having a system of generators $\alpha_j$ for $B$ as an $A$-module implies that we also have a system of generators for $L$ as a $K$-module. In fact, we have $dB=A\beta_1\oplus\cdots\oplus A\beta_m$ with $\beta_j\in dB$, hence $\frac{\beta_j}{d}\in B$. Now let $b\in B$, so that
	\begin{align*}
		db &= b_1\beta_1 + \cdots + b_m\beta_m\\
		b &= b_1\frac{\beta_1}{d} + \cdots + b_m\frac{\beta_m}{d}
	\end{align*}
	with $b_j\in A$. This shows that the rank of $B$ is also $m$, i.e. $\rank(B)\leq[L:K]$. The same argument as in the previous lemma shows that a system of the same length generates $L$ as a $K$-module, which forces $\rank(B)=[L:K]$. Now choose a system $\mu_1,\dots,\mu_r\in M$ of generators of the $B$-module $M$. Again, we can choose some $a_i\in A$ for which $a_i\mu_i\in B$; in fact, letting $a=a_1\cdots a_n$ we get a single element for which $a\mu_i\in B$ for all $i$. Hence $aM\subseteq B$, so that
	\[
		adM \subseteq dB \subseteq M_0
	\]
	with $M_0=A\alpha_1\oplus\cdots\oplus A\alpha_n$. By the same classification theorem, $M_0$ being a free $A$-module implies that $adM$ is one as well, hence also $M$. Putting everything together, we get
	\begin{align*}
		[L:K] = \rank(B) &\leq \rank(M)\\
			&= \rank(adM)\\
			&\leq \rank(M_0) = [L:K],
	\end{align*}
	that is $\rank(M)=[L:K]$.
\end{proof}

Looking back to the usual integral closure $\calO_K\subset K$ of $\bbZ\subset\bbQ$ in some algebraic number field $K$, the immediate application of these results on integral bases gives every finitely generated $\calO_K$-submodule $\fraka$ of $K$ a $\bbZ$-basis $\alpha_1,\dots,\alpha_n$, i.e. $\fraka = \alpha_1\bbZ+\cdots+\alpha_n\bbZ$. Since the discriminant $d(\alpha_1,\dots,\alpha_n)$ is independent of our choice of $\bbZ$-basis, we can simply write $d(\fraka)=d(\alpha_1,\dots,\alpha_n)$.

In particular, given an integral basis $\omega_1,\dots,\omega_n$ of $\calO_K$, we define the \emph{discriminant of the algebraic number field $K$} as
\[
	d_K = d(\calO_K) = d(\omega_1,\dots,\omega_n).
\]


The following proposition that relates two $\calO_K$-submodules will be useful later.

\begin{prop}
	If $\fraka\subseteq\fraka'$ are two non-zero finitely generated $\calO_K$-submodules of $K$, then the index $(\fraka:\fraka')$ is finite and satisfies
	\[
		d(\fraka) = (\fraka:\fraka')^2 d(\fraka').
	\]
\end{prop}
\begin{proof}
	As $\bbZ$-modules, $\fraka$ and $\fraka'$ have the same rank (by classification theorem). Let $A$ denote the change of basis matrix from a $\bbZ$-basis $\calB$ of $\fraka'$ to a $\bbZ$-basis $\calC$ of $\fraka$. By the Smith normal form, we get invertible matrices $P$ and $Q$ such that $PAQ$ is diagonal. Write $\calB'=Q\inv\calB$ and $\calC'=P\calC$, bases of $\fraka'$ and $\fraka$ respectively. It follows that $\calB'$ and $\calC'$ are stacked, so
	\[
		\fraka'/\fraka \cong \bbZ_{d_1} \oplus \cdots \oplus \bbZ_{d_n},
	\]
	with $d_i$ the diagonal entries of $PAQ$, hence $|\det(A)|=d_1\cdots d_n=(\fraka:\fraka')$. Therefore
	\begin{align*}
		d(\fraka) &= \det(A)^2 d(\fraka')\\
			&= (\fraka:\fraka')^2 d(\fraka').
	\end{align*}
\end{proof}


\section{Ideals}

This section culminates in the definition and finiteness result of the `class group' of an algebraic number field.


\section{Minkowski theory}

A good amount of fundamental questions in number theory (including the finiteness result of the class group) can be answered through the study of certain geometric aspects of number fields. To be more specific, one considers points of an $n$-dimensional algebraic number field $K\mid\bbQ$ as points in $n$-dimensional complex space.

\subsection{Lattices}

\begin{defn}
	Let $V$ be an $n$-dimensional $\bbR$-vector space. A \emph{lattice} in $V$ is a subgroup of the form
	\[
		\Gamma = \bbZ v_1 + \cdots + \bbZ v_m
	\]
	with linearly independent vectors $v_1,\dots,v_m\in V$. The $m$-tuple $(v_1,\dots,v_m)$ is called a \emph{basis} for the lattice and the set
	\[
		\Phi = \{x_1v_1 + \cdots + x_mv_m \mid x_i\in\bbR, 0\leq x_i< 1\}
	\]
	a \emph{fundamental mesh} of the lattice. It is called \emph{complete} or a \emph{$\bbZ$-structure} of $V$ if $m=n$.
\end{defn}

Note that a complete lattice is `complete' in the sense that translates of $\Phi$ cover all of $V$.

For our applications to number fields, we prefer to characterize lattices without referring to `linearly independent' vectors. To begin with, lattices are finitely generated subgroups of $V$, but not necessarily the other way around (e.g. $\bbZ+\bbZ\sqrt{2}$ is a subgroup of $V=\bbR$ but not a lattice, since 1 and $\sqrt{2}$ are not linearly independent). The full characterization considers lattices $\Gamma=\bbZ v_1 + \cdots + \bbZ v_m$ as \emph{discrete subgroups} of $V$: every point $\gamma\in\Gamma$ must have a neighborhood in which no other point of the lattice is contained, i.e. it must be isolated.

\begin{prop}
	A subgroup $\Gamma\subseteq V$ is a lattice if and only if it is discrete.
\end{prop}
\begin{proof}
	($\Longrightarrow$) Fix $\gamma\in\Gamma$, so that $\gamma=a_1v_1+\cdots+a_mv_m$ with $a_i\in\bbZ$. Extend the basis of $\Gamma$ to a basis of $V$; then the set
	\[
		\{x_1v_1 + \cdots + x_nv_n \mid x_i\in\bbR, |x_i-a_i|<1\}
	\]
	is a neighborhood in which $\gamma$ is isolated. Hence $\Gamma$ is discrete.

	($\Longleftarrow$) We first show that $\Gamma$ being a discrete subgroup of $V$ implies that it is closed: for choose any neighborhood $U$ of 0, then there is a smaller neighborhood $U'\subseteq U$ of 0 such that any difference of vectors in $U'$ is in $U$ (just make $U'$ small enough). If there were some $x\notin\Gamma$ in the closure of $\Gamma$, then $x+U'$ would contain a difference $\gamma_1-\gamma_2\in x+U'$ of distinct lattice points $\gamma_1,\gamma_2\in\Gamma$, but then $0\neq \gamma_1-\gamma_2\in U$, so the arbitary neighborhood $U$ around 0 contains the lattice point $\gamma_1-\gamma_2$, i.e. 0 is not isolated, so $\Gamma$ is not a discrete subgroup of $V$, a contradiction. Hence $\Gamma$ is closed.

	We now show that $\Gamma$ can be given a basis in the lattice sense, with linearly independent vectors. Let $V_0$ the be the linear subspace of $V$ spanned by $\Gamma$, and let $m$ be its dimension. Choose a basis $u_1,\dots,u_m\in\Gamma$ for $V_0$, and consider the complete lattice $\Gamma_0 = \bbZ u_1 + \cdots + \bbZ u_m \subseteq\Gamma$ of $V_0$. We claim that $q=(\Gamma:\Gamma_0)$ is finite: let $\gamma_i\in\Gamma$ vary over a system of representatives of the cosets in $\Gamma/\Gamma_0$. Since $\Gamma_0$ is complete, translates of its fundamental mesh $\Phi_0$ cover $V_0$. Since $\Gamma\cap\overline{\Phi_0}$ is compact and discrete, it is finite, which implies that the number of representatives $\gamma_i$ is finite.

	For any coset $\gamma+\Gamma_0$ in the quotient we have $q(\gamma+\Gamma_0)=\Gamma_0$, but $q(\gamma+\Gamma_0)=q\gamma+\Gamma_0$, so $q\gamma\in\Gamma_0$. The choice of $\gamma\in\Gamma$ was arbitrary, so $q\Gamma\subseteq\Gamma_0$. Hence
	\[
		\Gamma\subseteq\frac{1}{q}\Gamma_0 = \bbZ\left(\frac{1}{q}u_1\right) + \cdots + \bbZ\left(\frac{1}{q}u_m\right).
	\]

	By classification of finitely generated abelian groups, $\Gamma$ admits a $\bbZ$-basis $\{v_i\}$ of dimension $r\leq m$. But recall that $\Gamma$ spans $V_0$, which is $m$-dimensional, so $r=m$ and the $v_i$ are linearly independent.
\end{proof}

The following lemma is a criterion for identifying complete lattices.

\begin{lem}
	A lattice $\Gamma$ in $V$ is complete if and only if there exists a bounded subset $M\subseteq V$ such that the collection of all translates $M+\gamma$, with $\gamma\in\Gamma$, covers all of $V$.
\end{lem}
\begin{proof}
	($\Longrightarrow$) Given that $\Gamma=\bbZ v_1 + \cdots + \bbZ v_n$ is complete, we let $M$ be the fundamental mesh of the lattice. Translates of this, by definition, covers $V$.

	($\Longleftarrow$) Given such a bounded $M\subset V$, whose translates cover $V$, let $V_0$ be the subspace spanned by $\Gamma$. Let $v\in V$, so that for any $n\in\bbN$ we have
	\[
		nv = a_n + \gamma_n
	\]
	with $a_n\in M$ and $\gamma_n\in\Gamma\subset V_0$.

	By $M$ being bounded, $\frac{1}{n}a_n \to 0$ as $n\to\infty$, and by $V_0$ being closed we get
	\begin{align*}
		v &= \lim_{n\to\infty}\frac{1}{n}a_n + \lim_{n\to\infty}\frac{1}{n}\gamma_n\\
			&= \lim_{n\to\infty} \frac{1}{n}\gamma_n,
	\end{align*}
	which lies in $V_0$.
\end{proof}

If we restrict our attention to $V$ an euclidean vector space, we have a sensible notion of volume, which we denote by $\vol$.

A subset $X$ of $V$ is \emph{centrally symmetric} if, given any $x\in X$, the point $-x$ is also in $X$. The subset is \emph{convex} if the line segment joining two points $x,y\in X$ is contained in $X$.

\begin{thm}[Minkowski's Lattice Point Theorem]
	Let $\Gamma$ be a complete lattice in the euclidean vector space $V$ and $X$ a centrally symmetric, convex subset of $V$. Suppose that $\vol(X) > 2^n\vol(\Gamma)$. Then $X$ contains at least one non-zero lattice point of $\Gamma$.
\end{thm}
\begin{proof}
	If we find two distinct lattice points $\gamma_1,\gamma_2\in\Gamma$ such that
	\[
		\left(\frac{1}{2}X + \gamma_1\right)\cap\left(\frac{1}{2}X + \gamma_2\right) \neq \emptyset,
	\]
	we get a pair of points $x_1,x_2\in X$ such that
	\[
		\left(\frac{1}{2}x_1 + \gamma_1\right) = \left(\frac{1}{2}x_2 + \gamma_2\right),
	\]
	so $\gamma=\gamma_1-\gamma_2=\frac{1}{2}x_1-\frac{1}{2}x_2$ is a lattice point, and the midpoint between $x_1$ and $-x_2$ (which is in $X$ by central symmetry), so $\gamma\in X$ by $X$ being convex. Hence $\gamma\in\Gamma\cap X$.


	Assume that the two sets above were disjoint for any pair of lattice points; then $\Phi\cap\left(\frac{1}{2}X+\gamma\right)$ are also pairwise disjoint for any fundamental mesh $\Phi$, so 
	\[
		\vol(\Phi) \geq \sum_{\gamma\in\Gamma} \vol\left(\Phi\cap\left(\frac{1}{2}X+\gamma\right)\right).
	\]

	But translating $\Phi\cap\left(\frac{1}{2}X+\gamma\right)$ by $-\gamma$ gives $(\Phi-\gamma)\cap\frac{1}{2}X$, which has equal volume, and $\Phi-\gamma$ ranging over $\Gamma$ cover all of $V$, hence also $\frac{1}{2}X$. Therefore
	\begin{align*}
		\vol(\Phi) \geq \sum_{\gamma\in\Gamma} \vol\left(\Phi\cap\left(\frac{1}{2}X+\gamma\right)\right) &= \sum_{\gamma\in\Gamma}\vol\left((\Phi-\gamma)\cap\frac{1}{2}X\right)\\
			&= \vol\left(\frac{1}{2}X\right)\\
			&= \frac{1}{2^n}\vol(X),
	\end{align*}
	a contradiction.
\end{proof}

In other words; a `nice' (centrally symmetric, convex), `large enough' (volume condition) subset of $V$ always contains a lattice point.


\subsection{Setup in number fields}

In a nutshell, we will construct a vector space $K_\bbR$ from an arbitrary number field $K\mid\bbQ$ that will turn out to be isomorphic to a real euclidean vector space. Minkowski's lattice point theorem will apply, which will prove useful as a tool in our final result.

We start by defining the $\bbC$-vector space $K_\bbC := \prod_\tau \bbC$ as the product of the complex plane indexed by all $n$ complex embeddings $\tau:K\to\bbC$, and the canonical map $j:K\to K_\bbC$ sending each $a\in K$ to the vector $(\tau a)$.

Remember that the Galois group $G(\bbC\mid\bbR)$ is generated by complex conugation $F:z\mapsto\overline{z}$. This conjugation map acts on factors of $\prod_\tau\bbC$ and on the indexes themselves, by taking $\tau\mapsto\overline{\tau}$. In particular, we get the involution $F:K_\bbC\to K_\bbC$ that takes points $z=(z_\tau)\in K_\bbC$ to $(Fz)_\tau = \overline{z}_{\overline{\tau}}$.

Finally, we define a trace $\Tr:K_\bbC\to\bbC$ by $\Tr(z) = \sum_\tau z_\tau$, which is again equivariant under $F$. Note that the composition of maps
\begin{center}
	\begin{tikzcd}
		K \arrow[r,"j"] & K_\bbC \arrow[r,"\Tr"] & \bbC
	\end{tikzcd}
\end{center}

gives $\Tr\circ j(a) = \Tr((\tau a)_\tau) = \sum_\tau \tau a$, the usual trace $\Tr_{\bbC\mid\bbQ}$.

We now restrict our attention to the real vector space $K_\bbR = K_\bbC^+ = \left[\prod_\tau\bbC\right]^+$, consisting of the $G(\bbC\mid\bbR)$-invariant, i.e. conjugation-invariant points of $K_\bbC$; that is, $(z_\tau)\in K_\bbC$ for which $z_{\overline{\tau}}=\overline{z}_\tau$. Indeed, for such a point we have $(\overline{z})_{\overline{\tau}}= \overline{\overline{z}}_\tau = z_\tau$. Hence for $a\in K$, we have $F(ja)=ja$, so $j$ restricts to
\[
	j : K\to K_\bbR.
\]

We call the euclidean space $K_\bbR$ the \emph{Minkowski space} of $K$, its scalar product the \emph{canonical metric}, and the associated Haar measure (volume function) the \emph{canonical measure}. The trace restricts as well to the usual trace of $K\mid\bbQ$.

Of all the embeddings $\tau:K\to\bbC$, some have a purely real image, which we denote by
\[
	\rho_1,\dots,\rho_r:K\to\bbR,
\]
and the complex ones are given in pairs (by conjugation),
\[
	\sigma_1,\overline{\sigma_1},\dots,\sigma_s,\overline{\sigma_s}:K\to\bbC.
\]

Hence, we get 
\[
	K_\bbR = \{(z_\tau)\in\prod_\tau\bbC \mid z_\rho\in\bbR, z_{\overline{\sigma}} = \overline{z}_\sigma\}.
\]


\begin{prop}
	There is an isomorphism
	\[
		f : K_\bbR \to \prod_\tau \bbR = \bbR^{r+2s}
	\]
	given by $(z_\tau)\mapsto(x_\tau)$, where
	\begin{align*}
		x_\rho = z_\rho, x_\sigma = \Re(z_\sigma), x_{\overline{\sigma}} = \Im(z_\sigma).
	\end{align*}
\end{prop}
\begin{proof}
	By our previous discussion, this map is an isomorphism.
\end{proof}

The two different measures (on $K_\bbR$ and $\bbR^{r+2s}$) differ by
\[
	\vol_{\text{canonical}}(X) = 2^s\vol_{\text{Lebesgue}}(f(X)).
\]


The map $j:K\to K_\bbR$, which `embeds' the number field $K$ into the euclidean vector space $K_\bbR$, is the beginning of our study of `geometric' number theory.

\begin{prop}
	If $\fraka\neq 0$ is an ideal of $\calO_K$, then $\Gamma=j\fraka$ is a complete lattice in $K_\bbR$. Its fundamental mesh has volume
	\[
		\vol(\Gamma) = \sqrt{|d_K|}(\calO_K:\fraka).
	\]
\end{prop}
\begin{proof}
	Let $\alpha_1,\dots,\alpha_n$ be a $\bbZ$-basis for $\fraka$, i.e.
	\begin{align*}
		\Gamma = j\fraka &= j\left(\bbZ\alpha_1+\cdots+\bbZ\alpha_n\right)\\
			&= \bbZ j\alpha_1 + \cdots + \bbZ j\alpha_n,
	\end{align*}
	so $j\fraka$ is a lattice in $K_\bbR$. To see that it is complete, notice that

	Enumerate all the embeddings $\tau_i:K\to\bbC$, and we have a matrix $A=(\tau_i\alpha_j)$, which relates to the discriminant of $\fraka$ by
	\begin{align*}
		d(\fraka) = d(\alpha_1,\dots,\alpha_n) &= (\det A)^2\\
			&= (\calO_K:\fraka)^2 d_K,
	\end{align*}
	and
	\begin{align*}
		(\langle j\alpha_i, j\alpha_k\rangle) &= (\sum_{l=1}^n \tau_l\alpha_i \overline{\tau}_l\alpha_k)\\
			&= A\overline{A}^t,
	\end{align*}
	so 
	\begin{align*}
		\vol(\Gamma) &= \left|\det(\langle j\alpha_i,j\alpha_k\rangle)\right|^{1/2}\\
			&= \sqrt{|d_K|}(\calO_K:\fraka).
	\end{align*}
\end{proof}

The following result, chiefly using Minkowski's lattice theorem, is the fundamental tool in our next results on $Cl_K$.

\begin{thm}
	Let $\fraka\neq 0$ be an integral ideal of $K$, and let $c_\tau>0$, for $\tau\in\Hom(K,\bbC)$, be real numbers such that $c_\tau=c_{\overline{\tau}}$ and
	\[
		\prod_\tau c_\tau > A(\calO_k:\fraka),
	\]
	where $A=\left(\frac{2}{\pi}\right)^s\sqrt{|d_K|}$. Then there exists $a\in\fraka$, $a\neq 0$, such that
	\[
		|\tau a| < c_\tau
	\]
	for all $\tau\in\Hom(K,\bbC)$.
\end{thm}
\begin{proof}
	Let $X=\{(z_\tau)\in K_\bbR \mid |z_\tau| < c_\tau\}$. We claim that this set is centrally symmetric and convex. We bring $X$ over to euclidean space to compute its volume more easily, through the isomorphism we have defined in (??). Hence its volume is $2^s$ times the Lebesgue volume of the image
	\[
		f(X) = \{(x_\tau)\in\prod_\tau\bbR \mid |x_\rho| < c_\rho, x_{\sigma}^2+x_{\overline{\sigma}}^2 < c_{\sigma}^2\},
	\]
	i.e.
	\begin{align*}
		\vol(X) = 2^s \vol_{\text{Lebesgue}}(f(X)) &= 2^s\prod_\rho(2c_\rho)\prod_\sigma(\pi c_\sigma^2)\\
		&= 2^{r+s}\pi^s \prod_\tau c_\tau\\
		&> 2^{r+s}\pi^s \left(\frac{2}{\pi}\right)^s\sqrt{|d_K|}(\calO_K:\fraka)\\
		&= 2^{r+2s}\sqrt{|d_K|}(\calO_K:\fraka) = 2^n\vol(\Gamma),
	\end{align*}

	the last equality given by our previous proposition. Since $\Gamma$ is complete in $K_\bbR$ and $X$ centrally symmetric, convex and with large enough volume, our setup satisfies the conditions of the Minkowski lattice point theorem. This immediately gives a lattice point $\gamma$ in $X$, i.e. some $a\in\fraka$ for which $|\tau a|<c_\tau$ for all $\tau\in\Hom(K,\bbC)$.
\end{proof}



\subsection{Class number}

We want to show that $Cl_K=J_K/P_K$ is finite, i.e. that the number of classes of fractional ideals that differ by a multiple of a principel ideal is finite. Rough outline: show that under some sensible notion of the `norm' of an ideal, only finitely many ideals of $\calO_K$ are bounded above by some number. Then show that each class of $Cl_K$ has an integral ideal satisfying the same bound. Hence there are only finitely many classes, i.e. $Cl_K$ is finite.

Given an integral ideal $\fraka$, denote its \emph{absolute norm} by
\[
	\frakN(\fraka) = (\calO_K:\fraka).
\]

This index is again finite by our previous considerations. In particular, this quantity is deserves its name by considering the case $\fraka=(\alpha)$, for which
\[
	\frakN\left((\alpha)\right) = |N_{K\mid\bbQ}(\alpha)|,
\]
sine a $\bbZ$-basis $\omega_1,\dots,\omega_n$ of $\calO_K$ gives a $\bbZ$-basis $\alpha\omega_1,\dots,\alpha\omega_n$ of $(\alpha)=\alpha\calO_K$, and the change of basis matrix $A$ gives $|\det(A)|=(\calO_K:(\alpha))$, but $\det(A)=N_{K\mid\bbQ}(\alpha)$ too by definition.

We now show that the absolute norm is multiplicative over prime ideals.

\begin{prop}
	If $\fraka=\frakp_1^{v_1}\cdots\frakp_r^{v_r}$ is the prime factorization of an ideal $\fraka\neq 0$, then we have
	\[
		\frakN(\fraka) = \frakN(\frakp_1)^{v_1}\cdots\frakN(\frakp_r)^{v_r}.
	\]
\end{prop}
\begin{proof}
	By the Chinese remainder theorem, we have
	\[
		\calO_k/\fraka = (\calO_k/\frakp_1)^{v_1}\oplus\cdots\oplus(\calO_k/\frakp_r)^{v_r}.
	\]
	
	Hence the absolute norm of any $\fraka$ is the product of powers of the norms of its prime factors. We thus consider only the case where $\fraka=\frakp^v$.

	Note that the chain
	\[
		\frakp \supseteq \frakp^2 \supseteq \cdots \supseteq \frakp^v
	\]
	is proper, in the sense that $\frakp^i\neq\frakp^{i+1}$, by unique factorization. Furthermore, each quotient $\frakp^i/\frakp^{i+1}$ is a 1-dimensional $\calO_K/\frakp$-vector space: fix $a\in\frakp^i\setminus \frakp^{i+1}$ and let $\frakb=(a)+\frakp^{i+1}$. Then $\frakp^i\supseteq\frakb\supset\frakp^{i+1}$, and since $\frakb\frakp^{-i}$ cannot be a proper divisor of $\frakp=\frakp^{i+1}\frakp^{-i}$, for $\frakp$ is prime, we get that $\frakb=\frakp^i$, i.e. $a\pmod{\frakp^{i+1}}$ is a basis of $\frakp^i/\frakp^{i+1}$ as a 1-dimensional $\calO_K/\frakp$-vector space. Hence $\calO_K/\frakp\cong\frakp^i/\frakp^{i+1}$, and thus
	\begin{align*}
		\frakN(\frakp^v) = (\calO_K : \frakp^v) &= (\calO_K:\frakp)(\frakp:\frakp^2)\cdots(\frakp^{v-1}:\frakp^v)\\
			&= (\calO_K:\frakp)^v\\
			&= \frakN(\frakp)^v.
	\end{align*}

	The first equality concering indices is given by the following inductive argument. Let $M=M_0\supseteq M_1\supseteq \cdots\supseteq M_v$ be a sequence of modules for which each quotient $M_i/M_{i+1}$ is simple (it contains no non-zero proper submodules), and assume that $M$ is Noetherian and Artinian. Such a sequence takes the name of a \emph{composition sequence}, and by $M$ being Artinian we must have a finite $v$ for which $M_v=0$. In fact, we have (Eisenbud, 2.13) that any composition sequence of $M$ must have the same $v$, which we call its \emph{length}. We claim that, in general,
	\[
		|M| = \prod_{i=1}^{v-1} |M_i/M_{i+1}|.
	\]

	In fact, by the usual quotient homomorphism $M_0\to M_0/M_1$ we have $|M_0|=|M_0/M_1|\cdot|M_1|$. Again $M_1\to M_1/M_2$ gives $|M_1|=|M_1/M_2|\cdot|M_2|$, so $|M_0|=|M_0/M_1|\cdot|M_1/M_2|\cdot|M_2|$. It is clear that our equality follows.

	Our inductive proof is as follows. For the base case, let $v=1$, and consider the quotient homomorphism $M_0\twoheadrightarrow M_0/M_1=M_0$: this trivially gives $|M|=|M_0|$. Now assume that the product equality holds for modules of length up to $v-1$, and let $M$ be a module of length $v$. Any chain $M=M_0\supseteq M_1\supseteq\cdots\supseteq M_v$ quotiented out by $M_v$ gives a module of length $v-1$, for which the theorem holds. Lifting to our original composition sequence, (...)
	
	For this particular setup, we have $M_i/M_{i+1}\cong\frakp^i/\frakp^{i+1}\cong\calO_K/\frakp$.
\end{proof}

This automatically gives that the absolute norm is multiplicative over all ideals, and thus can be extended to a homomorphism
\[
	\frakN : J_K \to \bbR^*_+.
\]

\begin{lem}
	In every ideal $\fraka\neq 0$ of $\calO_K$ there exists a non-zero $a\in\fraka$ such that
	\[
		|N_{K\mid\bbQ}(a)| \leq \left(\frac{2}{\pi}\right)^s\sqrt{|d_K|}\frakN(\fraka).
	\]
\end{lem}
\begin{proof}
	Fix $\epsilon>0$, and choose positive real numbers $c_\tau$ for each $\tau\in\Hom(K,\bbC)$ such that $c_\tau = c_{\overline{\tau}}$ and
	\[
		\prod_\tau c_\tau = \left(\frac{2}{\pi}\right)^s\sqrt{|d_K|}\frakN(\fraka)+ \epsilon.
	\]

	Applying (??), we get some $a\in\fraka$, non-zero, for which $|\tau a|<c_\tau$. Hence
	\begin{align*}
		|N_{K\mid\bbQ}(a)| = \prod_\tau |\tau a| &< \prod_\tau c_\tau\\
			&= \left(\frac{2}{\pi}\right)^s\sqrt{|d_K|}\frakN(\fraka)+ \epsilon
	\end{align*}
	Our choice of $\epsilon$ was arbitrary, so letting $\epsilon\to 0$ we get the desired inequality.
\end{proof}

\begin{thm}
	The ideal class group $Cl_K$ is finite. Its order $h_K=(J_K:P_K)$ is called the \emph{class number} of $K$.
\end{thm}
\begin{proof}
	Let $\frakp\neq 0$ be a prime ideal of $\calO_K$, and $\frakp\cap\bbZ=p\bbZ$. Then, as mentioned before, $\calO_K/\frakp$ is a finite field extension of $\bbZ/p\bbZ$ of degree, say, $f$. Hence $\frakN(\frakp)=p^f$. In the other direction, a fixed $p$ gives only finitely many ideals $\frakp$ such that $\frakp\cap\bbZ=p\bbZ$, for then $\frakp\mid(p)$ and $(p)$ has finitely many prime factors. Hence for any given bound $M$, since there are only finitely many primes $p_i\leq M$, we conclude that there are only finitely many prime ideals of absolute norm $\leq M$. This same result applies to arbitrary integral ideals, by unique factorization and $\frakN$ being multiplicative.

	We only need to show that each class $[\fraka]\in Cl_K$ contains an integral ideal $\fraka_1$ with
	\[
		\frakN(\fraka_1) \leq M = \left(\frac{2}{\pi}\right)^s\sqrt{|d_K|}.
	\]

	In fact, choose any representative $\fraka$ of the class $[\fraka]$, and fix $\gamma\in\calO_K$ non-zero such that $\frakb=\gamma\fraka\inv\subseteq\calO_K$. By our previous lemma, we have some non-zero $\alpha\in\frakb$ for which
	\begin{align*}
		|N_{K\mid\bbQ}(\alpha)| \frakN(\frakb)\inv \leq \left(\frac{2}{\pi}\right)^s\sqrt{|d_K|},
	\end{align*}
	but in particular
	\begin{align*}
		|N_{K\mid\bbQ}(\alpha)| \frakN(\frakb)\inv &= \frakN((\alpha))\frakN(\frakb\inv)\\
			&= \frakN((\alpha)\frakb\inv),
	\end{align*}
	so $\frakN(\alpha\frakb\inv)\leq M$. Letting $\fraka_1=\alpha\frakb\inv=\alpha\alpha\inv\fraka$, we have found an ideal in $[\fraka]$ for which the desired inequality holds. By our previous reasoning there are only finitely many such ideals, and so $Cl_K$ is finite.
\end{proof}


\section{Units of a number field}

To use Minkowsky-theory results to talk about units in $\calO_K$, we need to translate our arguments to a multiplicative setting.

\subsection{Taking logarithms}

Recall the canonical mapping $j:K\to K_\bbC$ taking $a\in K$ to the $(r+2s)$-dimensional vector $(\tau a)$. Note that if we restrict the domain to $K^*$ (the unit field of $K$), it is clear that embeddings into $\bbC$ map units to units, so we get a homomorphism
\[
	j : K^* \to K^*_\bbC.
\]

Following a similar construction to that of the trace, we define the `norm' homomorphism
\[
	\N : K^*_\bbC \to \bbC^*
\]
given by multiplying the coordinates. Again, taking the composite
\begin{center}
	\begin{tikzcd}
		K^* \arrow[r,"j"] & K^*_\bbC \arrow[r,"\N"] & \bbC^*
	\end{tikzcd}
\end{center}
gives the usual norm of $K\mid\bbQ$, i.e. $\N_{K\mid\bbQ}(a)=\N(ja)$. The following diagram represents the bulk of our work:
\begin{center}
	\begin{tikzcd}
		K^* \arrow[d,"\N_{K\mid\bbQ}"'] \arrow[r,"j"] & K^*_\bbC \arrow[d,"\N"] \arrow[r,"l"] & \prod_\tau \bbR \arrow[d,"\Tr"]\\
		\bbQ^* \arrow[r,"\subset"] & \bbC^* \arrow[r,"l"] & \bbR.
	\end{tikzcd}
\end{center}

The most logical choice of a homomorphism from $K^*_\bbC$ (a multiplicative group) to $\prod_\tau\bbR$ (an additive group) is to apply the logarithm. We define $l:\bbC^*\to\bbR$ by $l(z)=\log|z|$, and this of course extends to a surjective $l:K^*_\bbC\to\prod_\tau\bbR$ by applying the logarithm on each coordinate. It is routine and requires no tricks to check that the diagram commutes.

Now, recall that the involution $F$, an element of the Galois group $G(\bbC\mid\bbR)$, acts trivially on $K^*$, on $K^*_\bbC$ as we defined in the previous section, and on any $(x_\tau)\in\prod_\tau\bbR$ by $(Fx)_\tau = x_{\overline{\tau}}$. As before, we restrict our attention to $G(\bbC\mid\bbR)$-invariant subgroups of the diagram above, and we get the following diagram:

\begin{center}
	\begin{tikzcd}
		K^* \arrow[d,"\N_{K\mid\bbQ}"'] \arrow[r,"j"] & K^*_\bbC \arrow[d,"\N"] \arrow[r,"l"] & \left[\prod_\tau \bbR\right]^+ \arrow[d,"\Tr"]\\
		\bbQ^* \arrow[r,"\subset"] & \bbR^* \arrow[r,"l"] & \bbR.
	\end{tikzcd}
\end{center}

\subsection{Dirichlet's unit theorem}

First of all, we can be sure that $\calO^*_K$ contains the group of roots of unity in $K$, which is of course finite; we denote it by $\mu(K)$. Indeed, if $a\in K$ is a root of unity, its norm $\N_{K\mid\bbQ}(a)$ is the product of all the roots of unity, which by Vieta equals $\pm 1$ times the constant term of $x^n-1=0$. Hence $\N_{K\mid\bbQ}(a)=\pm 1$, so $a$ is a unit. It is clear that roots of unity, by the polynomial they solve, are integral over $K$ and thus $\mu(K)$ lies in $\calO^*_K$. This is all we can say about the units of a number field thus far.

Consider the following subgroups of the upper part of the previous diagram:
\begin{itemize}
	\item $\calO^*_K = \{\epsilon\in\calO_K \mid \N_{K\mid\bbQ}(\epsilon) = \pm 1\}$, the group of units of $\calO_K$,
	\item $S=\{y\in K^*_\bbR \mid \N(y)=\pm 1\}$, the ``norm-one surface'', and
	\item $H=\{x\in\left[\prod_\tau\bbR\right]^+ \mid \Tr(x)=0\}$, the ``trace-zero hyperplane''.
\end{itemize}

These restrictions give homomorphisms
\begin{center}
	\begin{tikzcd}
		\calO^*_K \arrow[r,"j"] & S \arrow[r,"l"] & H,
	\end{tikzcd}
\end{center}
the composite of which, $\lambda := l\circ j$, gives an image that we denote by
\[
	\Gamma = \lambda(\calO_K) \subseteq H.
\]

\begin{prop}[Neukirch 7.1]
The sequence
\begin{center}
	\begin{tikzcd}
		1 \arrow[r] & \mu(K) \arrow[r,"\subset"] & \calO^*_K \arrow[r,"\lambda"] & \Gamma \arrow[r] & 0
	\end{tikzcd}
\end{center}
is exact.
\end{prop}
\begin{proof}
	Fix $\xi\in\mu(K)$. We have
	\begin{align*}
		\lambda(\xi) = l\circ j(\xi) &= \left(\log|\tau\xi|\right)_\tau\\
			&= (\log 1) = 0,
	\end{align*}
	so $\mu(K)\subseteq\ker\lambda$.

	Now fix $\epsilon\in\calO^*_K$. We have $\lambda(\epsilon)=0$, so $\log|\tau\epsilon|=0$ for each embedding $\tau$, i.e. $|\tau\epsilon|=1$. This implies that $j\epsilon$ is bounded in $K_\bbR$, but bounded regions of the lattice $j\calO_K$ contain finitely many lattice points, so $\ker\lambda$ is finite. Since it also contains $\mu(K)$ and is a group under multiplication, it is clear that $\epsilon$ is necessarily a root of unity.

	Hence $\ker\lambda\subseteq\mu(K)$, and thus $\ker\lambda=\mu(K)$.
\end{proof}

The following finiteness result will allow us to show that $\Gamma$ is a (complete) lattice in $H$.

\begin{lem}[Neukirch 7.2]
	Up to multiplication by units, there are only finitely many elements $\alpha\in\calO_K$ of given norm $\N_{K\mid\bbQ}(\alpha)=a$.
\end{lem}
\begin{proof}
	Fix $a\in\bbZ$, $a>1$. We claim that in every coset of of $\calO_K/a\calO_K$ there is at most (up to units) one element $\alpha$ with $|\N(\alpha)|=|\N_{K\mid\bbQ}(\alpha)|=a$. Else, if $\beta=\alpha+a\gamma$ is another ($\gamma\in\calO_K$), then
	\[
		\alpha/\beta = 1 \pm \N(\beta)/\beta \gamma \in \calO_K
	\]
	since $\N(\beta)/\beta \in \calO_K$, and similarly $\beta/\alpha\in\calO_K$, so $\beta$ is associated to $\alpha$.
\end{proof}

\begin{thm}[Neukirch 7.3]
	The group $\Gamma$ is a complete lattice in the $(r+s-1)$-dimensional vector space $H$, and is therefore isomorphic to $\bbZ^{r+s-1}$.
\end{thm}
\begin{proof}
	Neukirch's proof proceeds as follows: first, it is shown that $\Gamma$ is indeed a lattice by considering the mapping from which $\lambda:\calO_K^*\to H$ arises by restriction. To show that $\Gamma$ is complete, the proof invokes Minkowski's lattice theorem to construct a bounded set that spans the lattice (in this case, through multiplicative translates).
\end{proof}

Having established that $\Gamma$ has a nice $\bbZ$-module structure, understanding $\calO^*_K$ requires very little effort.

\begin{thm}[Dirichlet's unit theorem -- Neukirch 7.4]
	The group of units $\calO^*_K$ of $\calO_K$ is the direct product of the finite cyclic group $\mu(K)$ and a free abelian group of rank $r+s-1$.
\end{thm}
\begin{proof}
	Recall the exact sequence
	\begin{center}
		\begin{tikzcd}
			1 \arrow[r] & \mu(K) \arrow[r,"\subset"] & \calO^*_K \arrow[r,"\lambda"] & \Gamma \arrow[r] & 0.
		\end{tikzcd}
	\end{center}
	Let $v_1,\dots,v_t$ be a $\bbZ$-basis of $\Gamma$, and $\epsilon_1,\dots,\epsilon_t\in\calO^*_K$ be preimages of the $v_i$. Denote by $A$ the subgroup of $\calO^*_K$ generated by the $\epsilon_i$, and note that $\mu(K)\cap A=\{1\}$, for $\lambda(\epsilon_i)\neq 0$ except for the trivial case. Hence $A$ maps isomorphically onto $\Gamma$ through $\lambda$, so $\calO^*_K=\mu(K)\times A$.

	It is clear that $A$ is of rank $r+s-1$ over $\calO_K$.

\end{proof}

All this work shows that any unit $\epsilon\in\calO^*_K$ can be written uniquely as
\[
	\epsilon = \xi \epsilon_1^{v_1}\cdots\epsilon_t^{v_t},
\]
where $\xi$ is a root of unity and $\epsilon_i$ the ``fundamental'' units of $A$.


\subsection{Example -- Pell's equation}

\begin{prop}[Neukirch, Exercise 5.1]
	Let $D>1$ be a square-free integer and $d$ the discriminant of the real quadratic number field $K=\bbQ(\sqrt{D}))$. Let $x_1,y_1$ be the uniquely determined rational integer solution of one of the equations
	\[
		x^2-dy^2=\pm 4,
	\]
	with $x_1,y_1$ positive and as small as possible. Then
	\[
		\epsilon_1 = \frac{x_1+y_1\sqrt{d}}{2}
	\]
	is a fundamental unit of $K$.
\end{prop}
\begin{proof}
	First, note that $\{1,\sqrt{d}\}$ serves as a basis for $\bbQ(\sqrt{D})$ as a $\bbQ$-vector space. The key behind attacking Pell's equation lies in the equality
	\[
		x^2-dy^2 = (x+y\sqrt{d})(x-y\sqrt{d}) = \N_{K\mid\bbQ}(x+y\sqrt{d}).
	\]
	The goal is to show that an integral solution $(x_1,y_1)$ to Pell's equation does indeed give rise to an algebraic integer of the form of $\epsilon_1$, which is furthermore a unit, moreso \emph{fundamental}. The middle step is the easiest part of the proof; the rest relies in case-work depending on the residue of $D$ modulo 4. It is significant, and once more confirmation of the complexity of integer rings (or even more fundamentally, of integers), that one cannot easily avoid case-work here.

	In what follows, we let $(x_1,y_1)$ be the smallest pair of positive integers solving
	\[
		x_1^2-dy_1^2 = \pm 4.
	\]

	Neukirch's exercise \S 5.2 gives integral bases for all quadratic extensions of $\bbQ$. We readily use this result. Let $d$ be the discriminant of $K=\bbQ(\sqrt{D})$.

	For $D\equiv 1 \pmod{4}$, we have $d=D$ and an integral basis $\{1,\frac{1}{2}(1+\sqrt{d})\}$. If $x_1$ is even, $dy_1^2$ has to be even for Pell's equation to hold, but $d$ is odd, so $y_1$ is even as well. If $x_1$ is odd, $dy_1^2$ has to be odd for the same reason. In any case, we conclude that $x_1$ and $y_1$ must have the same parity, so their difference $x_1-y_1$ must be even. Note that
	\begin{align*}
		\epsilon_1 = \frac{x_1+y_1\sqrt{d}}{2} &= \frac{x_1-y_1+y_1+y_1\sqrt{d}}{2}\\
			&= \frac{x_1-y_1}{2} + y_1\frac{1}{2}\left(1+\sqrt{d}\right),
	\end{align*}
	so $\epsilon_1\in\calO_K$.

	For $D\equiv 2,3 \pmod{4}$, we have $d=4D$ and an integral basis $\{1,\frac{1}{2}\sqrt{d}\}$. Substituting $d$ in Pell's equation, we conclude that $x_1$ is even. Hence
	\[
		\epsilon_1 = \frac{x_1+y_1\sqrt{d}}{2} = \frac{x_1}{2} + y_1\frac{\sqrt{d}}{2},
	\]
	so $\epsilon_1\in\calO_K$.

	In any case, to check the norm of $\epsilon_1$ we need not leave $K$:
	\begin{align*}
		\N_{K\mid\bbQ}(\epsilon_1) &= \N_{K\mid\bbQ}\left(\frac{1}{2}\right)\N_{K\mid\bbQ}\left(x_1+y_1\sqrt{d}\right)\\
			&= \frac{\pm 4}{4} = \pm 1,
	\end{align*}
	so $\epsilon_1$ is indeed a unit.

	We now invoke Dirichlet's unit theorem: it tells us that $\calO^*_K = \mu(K)\times \bbZ^n$, with $n=\#\text{real embeddings}+\#\text{complex embeddings}-1$, in this case $n=2+0-1=1$ so $\calO^*_K = \mu(K)\times\bbZ$, but $\bbQ(\sqrt{D})\subset \bbR$ so $\mu(K)=\{\pm 1\}$, which shows that $\epsilon_1$ is \emph{not} a root of unity (for $y_1>0$), and simplifies the free part of the unit group.

	For the final part, let $u=u_1+u_2\sqrt{d}$ be a fundamental unit of our ring of integers $\calO_K$ (i.e. a generator of $\calO^*_K$), so that $(u_1,u_2)$ is a solution to Pell's equation, and assume for the sake of contradiction that $\epsilon_1\neq u$, so in particular $\epsilon_1=u^k$ for some $k>0$. We will use a monotoneness result regarding powers of numbers of the form of $\epsilon_1$. Fix $a$ and $b$ positive integers, and let $\beta = \frac{a+b\sqrt{d}}{2} = \frac{a}{2} + \frac{b}{2}\sqrt{d}$. In general, we have
	\[
		\beta^2 = \frac{a^2+db^2+2ab\sqrt{d}}{4} = \frac{a^2+db^2}{4} + \frac{ab}{2}\sqrt{d}.
	\]

	In any congruence case, the first term of the power $\beta^2$ is strictly greater than the first term of $\beta$. From this, we conclude that the first term of $u$ is necessarily smaller than the first term of $\epsilon_1$, i.e. smaller than $x_1$, which contradicts the minimality assumption.

\end{proof}


\section*{Acknowledgments}
I am thankful to Owen Barrett for his guidance and patience during the entirety of the REU program; all the beautiful things in number theory that have been the subject of this paper would still be inaccessible to me without his many hours of help. Many thanks to Peter May for his unrelenting efforts at organizing this fantastic program.

\begin{thebibliography}{9}

\bibitem{lang}
S. Lang.
\emph{Algebra.}
Springer-Verlag. 2002.

\bibitem{neukirch}
J. Neukirch.
\emph{Algebraic Number Theory.}
Springer-Verlag. 1999.

\bibitem{rouseball}
W. W. Rouse Ball.
\emph{A Short Account of the History of Mathematics.}
Dover Publications. 1960.

\end{thebibliography}

\end{document}


