\section{Some basic algebra}


\subsection{Modules over a ring}

In this entire presentation, we restrict our attention to commutative rings with identity $1\neq 0$. For this introductory section, $R$ and $S$ denote such rings.

\begin{defn}[Lang III, \S 1]
	A \emph{left-module} $M$ over $R$, or a \emph{left $R$-module} $M$, is an abelian group, usually written additively, together with an operation of $R$ on $M$ such that, for all $a,b\in R$ and $x,y\in M$ we have
	\begin{align*}
		(a+b)x = ax+bx \quad \text{and} \quad a(x+y)=ax+ay.
	\end{align*}
\end{defn}

One can similarly define \emph{right $R$-modules}, but we will generally work with left $R$-modules, and simply call them \emph{modules}.

Given an $R$-module $M$, we say that $N$ is a \emph{submodule} of $M$, or an \emph{$M$-submodule}, if it is an additive subgroup of $M$. Note that $R$ is a module over itself.


\subsection{Polynomial rings}

For any commutative ring $R$ with identity, we define the ring of polynomials in the variable $x$ over $R$, denoted by $R[x]$, as all formal sums of the form
\[
	a_nx^n+a_{n-1}x^{n-1}+\cdots + a_1x + a_0,
\]
where $x$ is an `indeterminate', $a_nx^n$ the leading term, $a_n$ the leading coefficient, and $n$ the degree of the polynomial. A polynomial is \emph{monic} if $a_n=1$.

We define addition of polynomials component-wise, and multiplication by distributing over addition, defining $(ax^i)(bx^j)=abx^{i+j}$. These two operations make $R[x]$ into a ring.

$R$ can be naturally embedded into $R[x]$ by the constant polynomial map; this gives $R[x]$ an identity, and $R$ being commutative makes $R[x]$ commutative.



\subsection{Ring homomorphisms}

We often construct maps between rings. Frequently, the ring structure of the domain is preserved in the codomain through the map given.

\begin{defn}
	A \emph{ring homomorphism} is a map $\phi:R\to S$ of rings satisfying
	\begin{enumerate}[(i)]
		\item $\phi(a+b)=\phi(a)+\phi(b)$,
		\item $\phi(ab)=\phi(a)\phi(b)$ and
		\item $\phi(1_R) = 1_S$
	\end{enumerate}
	for all $a,b\in R$, where $1_R$ is the identity in $R$ and respectively for $S$.
\end{defn}

The \emph{kernel} of $\phi$ is defined by $\ker\phi=\phi\inv(0)$. This set motivates the notion of an \emph{ideal}, which is in turn a good example of a module over $R$. We now make this precise.

\begin{defn}
	An \emph{ideal} $I$ in a ring $R$ is an $R$-submodule over $R$.
\end{defn}

\begin{prop}
	For any ring homomorphism $\phi:R\to S$, its image is a subring of $S$ and its kernel is an ideal in $R$.
\end{prop}
\begin{proof}
	We only show the second part, which concerns the language of modules.

	First, if $a,b\in\ker\phi$, then
	\[
		\phi(a+b) = \phi(a)+\phi(b) = 0 + 0 = 0,
	\]
	so that $a+b\in\ker\phi$. Hence the kernel is a subgroup of $R$ under addition. Furthermore, given any $a\in\ker\phi$, the operation of $R$ on $\ker\phi$ given by multiplication has the following property:
	\[
		\phi(ar) = \phi(a)\phi(r) = 0\phi(r) = 0,
	\]
	so that $ar\in\ker\phi$, thus defining a valid action of $R$ on $\ker\phi$. Therefore $\ker\phi$ is a submodule of $R$, or an $R$-submodule: in other words, an ideal.
\end{proof}

\begin{defn}
	Let $I\subset R$ be an $R$-submodule. Define $rI=\{ra \mid a\in I\}$ and $Ir=\{ar \mid a\in I\}$ for any $r\in R$.
\end{defn}


Just like we do for groups, we can partition $R$ into \emph{cosets} $r+I$, and endow the set of cosets with the ring structure inherited from $R$: namely, defining $(r+I)+(s+I)=(r+s)+I$ and $(r+I)\cdot (s+I) = (r\cdot s) + I$. And crucially, this is only well-defined if $I$ is an ideal, not just a subset of $R$. The set of cosets under this ring structure is denoted by $R/I$ and called the \emph{quotient ring} of $R$ modulo $I$.

The group isomorphism theorems carry nicely to rings: every $R$-submodule is indeed the kernel of some ring homomorphism, etc. Within the richer ring structure, we can define addition and multiplication of ideals:

\begin{defn}
	Let $I$ and $J$ be ideals of $R$.
	\begin{enumerate}[(1)]
		\item Their sum is given by $I+J=\{a+b \mid a\in I, b\in J\}$.
		\item Their product is denoted by $IJ$ and defined as the set of all finite sums of pairs $ab$ with $a\in I$, $b\in J$.
	\end{enumerate}
\end{defn}

We restrict our attention to rings with identity $1\neq 0$.

\begin{defn}
	Given any subset $A$ of a ring $R$, we denote by $(A)$ the smallest ideal of $R$ containing $A$, and call it the \emph{ideal generated by $A$}.

	An ideal generated by a single element is called a \emph{principal ideal}.

	An ideal generated by a finite set is called a \emph{finitely generated ideal}.
\end{defn}

We can precisely obtain $(A)$ by taking all ideals $I$ such that $A\subset I$; their intersection is clearly not empty, and $(A)=\bigcap_{A\subset I} I$.

We also note that ideals truly capture in some sense the arithmetic notion of divisibility, familiar from $\bbZ$: an element $b\in R$ is in the ideal $(a)$ generated by some other $a\in R$ iff $b=ar$ for some $r\in R$, i.e. if $b$ is a \emph{multiple} of $a$, or equivalently if $a$ divides $b$. Furthermore, $b\in(a)$ iff $(b)\subset(a)$. We will make these notions more precise after restricting our attention to special rings later on.

\begin{defn}
	An ideal $\frakM\subset S$ of a ring $S$ is \emph{maximal} if $\frakM\neq S$ and the only ideals that contain $\frakM$ are $\frakM$ itself and the entire ring $S$.
\end{defn}

\begin{prop}
	For any commutative ring $R$, an ideal $\frakM$ is maximal iff $R/\frakM$ is a field.
\end{prop}





