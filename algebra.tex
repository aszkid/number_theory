\section{Some basic algebra}



\subsection{Polynomial rings}

For any commutative ring $R$ with identity, we define the ring of polynomials in the variable $x$ over $R$, denoted by $R[x]$, as all formal sums of the form
\[
	a_nx^n+a_{n-1}x^{n-1}+\cdots + a_1x + a_0,
\]
where $x$ is an `indeterminate', $a_nx^n$ the leading term, $a_n$ the leading coefficient, and $n$ the degree of the polynomial. A polynomial is monic if $a_n=1$.

We define addition of polynomials component-wise, and multiplication by distributivity defining $(ax^i)(bx^j)=abx^{i+j}$; this makes $R[x]$ into a ring

$R$ can be naturally embedded into $R[x]$ as the constant polynomials; this gives $R[x]$ an identity, and $R$ being commutative makes $R[x]$ commutative.



\subsection{Ring homomorphisms and ideals}

Ideals naturally arise as kernels of ring homomorphisms, just like normal subgroups spring from group homomorphisms.

\begin{defn}
	A \emph{ring homomorphism} is a map $\phi:R\to S$ of rings satisfying
	\begin{enumerate}[(i)]
		\item $\phi(a+b)=\phi(a)+\phi(b)$ and
		\item $\phi(ab)=\phi(a)\phi(b)$
	\end{enumerate}
	for all $a,b\in R$.

	The \emph{kernel} of $\phi$ is defined by $\ker\phi=\phi\inv(0)$.
\end{defn}

\begin{prop}
	For any ring homomorphsim $\phi:R\to S$, its image and kernel are subrings of $S$ and $R$ respectively.
\end{prop}

In fact, the kernel has a somewhat stronger property: for any $r\in R$ and $k\in\ker\phi$, we have that $\phi(rk)=\phi(r)\phi(k)=\phi(r)0=0$, so $rk,kr\in\ker\phi$. This is a normality property; in ring theoretic language, we say that $\ker\phi$ is an \emph{ideal} of $R$.

\begin{defn}
	Let $I\subset R$ be a subring. Define $rI=\{ra \mid a\in I\}$ and $Ir=\{ar \mid a\in I\}$ for any $r\in R$.

	We say that $I$ is an \emph{ideal} if we always have the inclusions $rI\subset I$ and $Ir\subset I$.
\end{defn}

(There is a general distinction between \emph{left-} and \emph{right-}ideals, but they agree for commutative rings. In particular, number fields and integral rings are commutative, so we are content with sticking to the commutative case.)

Just like we do for groups, we can partition $R$ in cosets $r+I$, and endow the set of cosets with the ring structure inherited from $R$: namely, defining $(r+I)+(s+I)=(r+s)+I$ and $(r+I)\cdot (s+I) = (r\cdot s) + I$. And again, this is only well-defined if $I$ is an ideal, not just a subring of $R$. The set of cosets under this ring structure is denoted by $R/I$ and called the \emph{quotient ring} of $R$ modulo $I$.

The group isomorphism theorems carry nicely to rings: every ideal of a ring is indeed the kernel of some ring homomorphism, etc.

\emph{Example;} consider $\bbZ[x]$, and let $I$ be the set of polynomials with degree $\geq 2$. It is clear that the product of any $p(x)\in\bbZ[x]$ with such a polynomial is at least of degree 2. Hence $I$ is an ideal, and we can form the quotient ring $\bbZ[x]/I$. Two polynomials $p(x),q(x)\in\bbZ[x]$ are identified in the quotient if $p(x)-q(x)=r(x)$ for some $r(x)\in I$; i.e. they differ by a degree $\geq 2$ polynomial, which is only the case if they agree on their degree 0 and 1 terms.

Hence $\overline{(1+3x)}+\overline{(-4+5x)}=\overline{-3+8x}$ and $\overline{(1+3x)}\overline{(-4+5x)}=\overline{-4+5x-12x+15x^2}=\overline{-4-7x}$.

Within the richer ring structure, we can define operations between ideals:

\begin{defn}
	Let $I$ and $J$ be ideals of $R$.
	\begin{enumerate}[(1)]
		\item Their sum is given by $I+J=\{a+b \mid a\in I, b\in J\}$.
		\item Their product is denoted by $IJ$ and defined as the set of all finite sums of pairs $ab$ with $a\in I$, $b\in J$.
	\end{enumerate}
\end{defn}

We restrict our attention to rings with identity $1\neq 0$.

\begin{defn}
	Given any subset $A$ of a ring $R$, we denote by $(A)$ the smallest ideal of $R$ containing $A$, and call it the \emph{ideal generated by $A$}.

	An ideal generated by a single element is called a \emph{principal ideal}.

	An ideal generated by a finite set is called a \emph{finitely generated ideal}.
\end{defn}

We can precisely obtain $(A)$ by taking all ideals $I$ such that $A\subset I$; since at least $R\subset I$, their intersection is not empty, so $(A)=\bigcap_{A\subset I} I$.

We also note that ideals truly capture in some sense the arithmetic notion of divisibility, familiar from $\bbZ$: an element $b\in R$ is in the ideal $(a)$ generated by some other $a\in R$ iff $b=ar$ for some $r\in R$, i.e. if $b$ is a \emph{multiple} of $a$, or equivalently if $a$ divides $b$. Furthermore, $b\in(a)$ iff $(b)\subset(a)$.

\begin{defn}
	An ideal $M\subset S$ of a ring $S$ is \emph{maximal} if $M\neq S$ and the only ideals that contain $M$ are $M$ itself and the entire ring $S$.
\end{defn}

\begin{prop}
	For any commutative ring $R$, an ideal $M$ is maximal iff $R/M$ is a field.
\end{prop}




\subsection{Fields and embeddings}







